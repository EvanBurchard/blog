\documentclass[]{article}
\usepackage{lmodern}
\usepackage{amssymb,amsmath}
\usepackage{ifxetex,ifluatex}
\usepackage{fixltx2e} % provides \textsubscript
\ifnum 0\ifxetex 1\fi\ifluatex 1\fi=0 % if pdftex
  \usepackage[T1]{fontenc}
  \usepackage[utf8]{inputenc}
\else % if luatex or xelatex
  \ifxetex
    \usepackage{mathspec}
    \usepackage{xltxtra,xunicode}
  \else
    \usepackage{fontspec}
  \fi
  \defaultfontfeatures{Mapping=tex-text,Scale=MatchLowercase}
  \newcommand{\euro}{€}
\fi
% use upquote if available, for straight quotes in verbatim environments
\IfFileExists{upquote.sty}{\usepackage{upquote}}{}
% use microtype if available
\IfFileExists{microtype.sty}{\usepackage{microtype}}{}
\usepackage[margin=1in]{geometry}
\usepackage{color}
\usepackage{fancyvrb}
\newcommand{\VerbBar}{|}
\newcommand{\VERB}{\Verb[commandchars=\\\{\}]}
\DefineVerbatimEnvironment{Highlighting}{Verbatim}{commandchars=\\\{\}}
% Add ',fontsize=\small' for more characters per line
\newenvironment{Shaded}{}{}
\newcommand{\KeywordTok}[1]{\textcolor[rgb]{0.00,0.44,0.13}{\textbf{{#1}}}}
\newcommand{\DataTypeTok}[1]{\textcolor[rgb]{0.56,0.13,0.00}{{#1}}}
\newcommand{\DecValTok}[1]{\textcolor[rgb]{0.25,0.63,0.44}{{#1}}}
\newcommand{\BaseNTok}[1]{\textcolor[rgb]{0.25,0.63,0.44}{{#1}}}
\newcommand{\FloatTok}[1]{\textcolor[rgb]{0.25,0.63,0.44}{{#1}}}
\newcommand{\ConstantTok}[1]{\textcolor[rgb]{0.53,0.00,0.00}{{#1}}}
\newcommand{\CharTok}[1]{\textcolor[rgb]{0.25,0.44,0.63}{{#1}}}
\newcommand{\SpecialCharTok}[1]{\textcolor[rgb]{0.25,0.44,0.63}{{#1}}}
\newcommand{\StringTok}[1]{\textcolor[rgb]{0.25,0.44,0.63}{{#1}}}
\newcommand{\VerbatimStringTok}[1]{\textcolor[rgb]{0.25,0.44,0.63}{{#1}}}
\newcommand{\SpecialStringTok}[1]{\textcolor[rgb]{0.73,0.40,0.53}{{#1}}}
\newcommand{\ImportTok}[1]{{#1}}
\newcommand{\CommentTok}[1]{\textcolor[rgb]{0.38,0.63,0.69}{\textit{{#1}}}}
\newcommand{\DocumentationTok}[1]{\textcolor[rgb]{0.73,0.13,0.13}{\textit{{#1}}}}
\newcommand{\AnnotationTok}[1]{\textcolor[rgb]{0.38,0.63,0.69}{\textbf{\textit{{#1}}}}}
\newcommand{\CommentVarTok}[1]{\textcolor[rgb]{0.38,0.63,0.69}{\textbf{\textit{{#1}}}}}
\newcommand{\OtherTok}[1]{\textcolor[rgb]{0.00,0.44,0.13}{{#1}}}
\newcommand{\FunctionTok}[1]{\textcolor[rgb]{0.02,0.16,0.49}{{#1}}}
\newcommand{\VariableTok}[1]{\textcolor[rgb]{0.10,0.09,0.49}{{#1}}}
\newcommand{\ControlFlowTok}[1]{\textcolor[rgb]{0.00,0.44,0.13}{\textbf{{#1}}}}
\newcommand{\OperatorTok}[1]{\textcolor[rgb]{0.40,0.40,0.40}{{#1}}}
\newcommand{\BuiltInTok}[1]{{#1}}
\newcommand{\ExtensionTok}[1]{{#1}}
\newcommand{\PreprocessorTok}[1]{\textcolor[rgb]{0.74,0.48,0.00}{{#1}}}
\newcommand{\AttributeTok}[1]{\textcolor[rgb]{0.49,0.56,0.16}{{#1}}}
\newcommand{\RegionMarkerTok}[1]{{#1}}
\newcommand{\InformationTok}[1]{\textcolor[rgb]{0.38,0.63,0.69}{\textbf{\textit{{#1}}}}}
\newcommand{\WarningTok}[1]{\textcolor[rgb]{0.38,0.63,0.69}{\textbf{\textit{{#1}}}}}
\newcommand{\AlertTok}[1]{\textcolor[rgb]{1.00,0.00,0.00}{\textbf{{#1}}}}
\newcommand{\ErrorTok}[1]{\textcolor[rgb]{1.00,0.00,0.00}{\textbf{{#1}}}}
\newcommand{\NormalTok}[1]{{#1}}
\ifxetex
  \usepackage[setpagesize=false, % page size defined by xetex
              unicode=false, % unicode breaks when used with xetex
              xetex]{hyperref}
\else
  \usepackage[unicode=true]{hyperref}
\fi
\hypersetup{breaklinks=true,
            bookmarks=true,
            pdfauthor={Justin Le},
            pdftitle={Auto: Building a declarative chat bot with implicit serialization},
            colorlinks=true,
            citecolor=blue,
            urlcolor=blue,
            linkcolor=magenta,
            pdfborder={0 0 0}}
\urlstyle{same}  % don't use monospace font for urls
% Make links footnotes instead of hotlinks:
\renewcommand{\href}[2]{#2\footnote{\url{#1}}}
\setlength{\parindent}{0pt}
\setlength{\parskip}{6pt plus 2pt minus 1pt}
\setlength{\emergencystretch}{3em}  % prevent overfull lines
\setcounter{secnumdepth}{0}

\title{Auto: Building a declarative chat bot with implicit serialization}
\author{Justin Le}
\date{February 29, 2016}

\begin{document}
\maketitle

\emph{Originally posted on
\textbf{\href{http://home.jle0.com:4111/entry/auto-building-a-declarative-chatbot-with-implicit-serialization.html}{in
Code}}.}

Today we're going to look at building a declarative chatbot using the
denotational components from the
\href{http://hackage.haskell.org/package/auto}{auto} library that is
modular and has implicit serialization. Most importantly, we'll look at
the ``design process'', and principles of architecture that you can
apply to your own projects.

\section{Overall Layout}\label{overall-layout}

\emph{auto} is a library that at the highest level gives you a stream
transformer. Transform a stream of inputs to a stream of outputs. So
when we approach a chat bot, we have to think --- what are the inputs,
and what are the outputs?

The choice should be pretty straightforward -- our input stream is a
stream of input messages from the irc server, and our output stream is a
stream of messages to send to the server. In haskell we like types, so
let's make some types.

\begin{Shaded}
\begin{Highlighting}[]
\CommentTok{-- source: https://github.com/mstksg/blog/tree/develop/code-samples/auto/chatbot.hs#L29-44}
\KeywordTok{type} \DataTypeTok{Nick}    \FunctionTok{=} \DataTypeTok{String}
\KeywordTok{type} \DataTypeTok{Channel} \FunctionTok{=} \DataTypeTok{String}
\KeywordTok{type} \DataTypeTok{Message} \FunctionTok{=} \DataTypeTok{String}

\KeywordTok{data} \DataTypeTok{InMessage} \FunctionTok{=} \DataTypeTok{InMessage} \NormalTok{\{ _}\OtherTok{inMessageNick   ::} \DataTypeTok{Nick}
                           \NormalTok{, _}\OtherTok{inMessageBody   ::} \DataTypeTok{Message}
                           \NormalTok{, _}\OtherTok{inMessageSource ::} \DataTypeTok{Channel}
                           \NormalTok{, _}\OtherTok{inMessageTime   ::} \DataTypeTok{UTCTime}
                           \NormalTok{\} }\KeywordTok{deriving} \DataTypeTok{Show}

\KeywordTok{newtype} \DataTypeTok{OutMessages} \FunctionTok{=} \DataTypeTok{OutMessages} \NormalTok{(}\DataTypeTok{Map} \DataTypeTok{Channel} \NormalTok{[}\DataTypeTok{Message}\NormalTok{]) }\KeywordTok{deriving} \DataTypeTok{Show}

\KeywordTok{instance} \DataTypeTok{Monoid} \DataTypeTok{OutMessages} \KeywordTok{where}
    \NormalTok{mempty  }\FunctionTok{=} \DataTypeTok{OutMessages} \NormalTok{M.empty}
    \NormalTok{mappend (}\DataTypeTok{OutMessages} \NormalTok{m1) (}\DataTypeTok{OutMessages} \NormalTok{m2)}
            \FunctionTok{=} \DataTypeTok{OutMessages} \NormalTok{(M.unionWith (}\FunctionTok{++}\NormalTok{) m1 m2)}
\end{Highlighting}
\end{Shaded}

We make some type aliases to make things a bit clearer. Our inputs are
going to be a data type/``struct'' with a nick, a body, a source, and a
time. Our outputs are going to be a \texttt{Data.Map.Map} from
\emph{containers} associating channels with messages to send. I'm just
adding here a \texttt{Monoid} instance in case we want to combine
\texttt{OutMessages} maps.

The type for a chat bot over a monad \texttt{m} would then be:

\begin{Shaded}
\begin{Highlighting}[]
\CommentTok{-- source: https://github.com/mstksg/blog/tree/develop/code-samples/auto/chatbot.hs#L46-46}
\KeywordTok{type} \DataTypeTok{ChatBot} \NormalTok{m }\FunctionTok{=} \DataTypeTok{Auto} \NormalTok{m }\DataTypeTok{InMessage} \DataTypeTok{OutMessages}
\end{Highlighting}
\end{Shaded}

A \texttt{ChatBot} takes a stream of \texttt{InMessage}s and returns a
stream of \texttt{OutMessages}s\ldots{}and might have effects in
\texttt{m} as it does so.

Note that we get a free instance of \texttt{Monoid} on
\texttt{ChatBot\ m}:

\begin{Shaded}
\begin{Highlighting}[]
\OtherTok{mappend ::} \DataTypeTok{ChatBot} \NormalTok{m }\OtherTok{->} \DataTypeTok{ChatBot} \NormalTok{m }\OtherTok{->} \DataTypeTok{ChatBot} \NormalTok{m}
\end{Highlighting}
\end{Shaded}

That takes two \texttt{ChatBot}s and creates a new \texttt{ChatBot} that
forks the input stream (sends all \texttt{InMessage}s) to both of the
original ones, and \texttt{mappend}s the results. So the new
\texttt{ChatBot} will send message to both original ones and return a
``combined'' \texttt{OutMessages}.

However, not all modules really have to ``care'' about the room of the
outputs\ldots{}they might just always reply directly to the room they
received the message on. So it'll help us to also make another sort of
\texttt{Auto}:

\begin{Shaded}
\begin{Highlighting}[]
\CommentTok{-- source: https://github.com/mstksg/blog/tree/develop/code-samples/auto/chatbot.hs#L47-47}
\KeywordTok{type} \DataTypeTok{RoomBot} \NormalTok{m }\FunctionTok{=} \DataTypeTok{Auto} \NormalTok{m }\DataTypeTok{InMessage} \NormalTok{(}\DataTypeTok{Blip} \NormalTok{[}\DataTypeTok{Message}\NormalTok{])}
\end{Highlighting}
\end{Shaded}

A \texttt{RoomBot} doesn't care where its messages go\ldots{}it just
replies to the same room it got its input from. It outputs a blip stream
of message lists; when it doesn't want to send messages out, it doesn't
emit. When it does, it \emph{does} emit, with the list of messages.

\subsection{Converting}\label{converting}

We can write a quick helper function to convert a \texttt{RoomBot} into
a full-on \texttt{ChatBot}, so we can merge them together with
\texttt{mappend}/\texttt{(\textless{}\textgreater{})}:

\begin{Shaded}
\begin{Highlighting}[]
\CommentTok{-- source: https://github.com/mstksg/blog/tree/develop/code-samples/auto/chatbot.hs#L49-52}
\OtherTok{perRoom ::} \DataTypeTok{Monad} \NormalTok{m }\OtherTok{=>} \DataTypeTok{RoomBot} \NormalTok{m }\OtherTok{->} \DataTypeTok{ChatBot} \NormalTok{m}
\NormalTok{perRoom rb }\FunctionTok{=} \NormalTok{proc inp }\OtherTok{->} \KeywordTok{do}
    \NormalTok{messages }\OtherTok{<-} \NormalTok{fromBlips [] }\FunctionTok{.} \NormalTok{rb }\FunctionTok{-<} \NormalTok{inp}
    \NormalTok{id }\FunctionTok{-<} \DataTypeTok{OutMessages} \FunctionTok{$} \NormalTok{M.singleton (_inMessageSource inp) messages}
\end{Highlighting}
\end{Shaded}

(This example uses proc notation; see this
\href{https://github.com/mstksg/auto/blob/master/tutorial/tutorial.md\#brief-primer-on-proc-notation}{proc
notation primer} for a quick run-down of the relevant aspects)

We say that \texttt{messages} is just the output of \texttt{rb} fed with
the input, except it ``collapses'' the blip stream into a normal stream
by substituting in \texttt{{[}{]}} whenever the stream doesn't emit. So
\texttt{messages} is \texttt{{[}{]}} when \texttt{rb} doesn't emit (it
doesn't want to send anything), and \texttt{messages} is
\texttt{{[}message1,\ message2\ ...{]}}, with the emitted contents, when
it \emph{does}.

The ``output'' will be a singleton map with the source of the input and
the messages to send to that source.

So now if we have a \texttt{RoomBot\ m}, we can convert it up into a
\texttt{ChatBot\ m}, and combine it/merge it with other
\texttt{ChatBot\ m}s.

\subsection{The whole deal}\label{the-whole-deal}

We have enough now then to imagine our entire program architecture:

\begin{itemize}
\tightlist
\item
  Write a bunch of separate modules, as \texttt{ChatBot\ m}s or
  \texttt{RoomBot\ m}s, which ever one is more convenient. The beauty is
  that we can merge them all together in the end with our promoter.
\item
  Combine all of our modules with \texttt{mconcat} --- that is,
  something like
  \texttt{chatBot\ =\ mconcat\ {[}module1,\ module2,\ module3,\ module\ 4{]}}.
  And that's it, that's our entire chat bot!
\item
  Having an overall \texttt{chatBot\ ::\ ChatBot\ m}, we can use
  something like \texttt{runOnChan} from \texttt{Control.Auto.Run} to
  have it exist on a concurrent thread and whatch a channel for input,
  and perform an action on output.
\item
  Find an out-of-the-box irc library that can trigger adding something
  to a concurrent queue when it receives a message, and where you can
  send messages to rooms.
\end{itemize}

And\ldots{}that's it. Program logic in our \texttt{ChatBot\ m}s, and
handling the ``view''/input with our backend.

\subsubsection{Free Serialization}\label{free-serialization}

Remember that \emph{auto} gives us the ability to serialize and resume
our \texttt{Auto}s for free\ldots{}so we can at any time save the state
of our chat bot to disk, and resume it when we re-load. We don't have to
worry about manually gathering our state between each \texttt{Auto} and
writing serialization code.

There's a ``convenience combinator'' called
\texttt{serializing\textquotesingle{}} in
\texttt{Control.Auto.Serialize} (it's one of many different ones that
can do something like this;
\href{http://hackage.haskell.org/package/auto/docs/Control-Auto-Serialize.html}{check
out the module} to see other ways of varying disciplined-ness!). It'll
take any \texttt{Auto} and turn it into an \texttt{Auto} that
``self-serializes'' --- when you begin running it, it automatically
loads its previous state if it exists, and as you run it, it
automatically maintains an updated ``resume state'' on disk.

\begin{Shaded}
\begin{Highlighting}[]
\OtherTok{serializing' ::} \DataTypeTok{MonadIO} \NormalTok{m }\OtherTok{=>} \NormalTok{FilePath }\OtherTok{->} \DataTypeTok{ChatBot} \NormalTok{m }\OtherTok{->} \DataTypeTok{ChatBot} \NormalTok{m}
\end{Highlighting}
\end{Shaded}

Note that
\texttt{serializing\textquotesingle{}\ fp\ ::\ MonadIO\ m\ =\textgreater{}\ ChatBot\ m\ -\textgreater{}\ ChatBot\ m}.
It looks a lot like an ``identity-ish'' sort of function, right? That's
because it is meant to behave \emph{like} \texttt{id}\ldots{}the
returned \texttt{ChatBot} behaves identical to the previous
one\ldots{}except it splices in the serializing action in-between. (We
are in \texttt{MonadIO} now, because the \texttt{Auto} has to access
\texttt{IO} in order to serialize itself between steps).

So, instead of

\begin{Shaded}
\begin{Highlighting}[]
\OtherTok{chatBot ::} \DataTypeTok{Monad} \NormalTok{m }\OtherTok{=>} \DataTypeTok{ChatBot} \NormalTok{m}
\NormalTok{chatBot }\FunctionTok{=} \NormalTok{mconcat [module1, module2, module3]}
\end{Highlighting}
\end{Shaded}

We can do:

\begin{Shaded}
\begin{Highlighting}[]
\OtherTok{chatBot ::} \DataTypeTok{MonadIO} \NormalTok{m }\OtherTok{=>} \DataTypeTok{ChatBot} \NormalTok{m}
\NormalTok{chatBot }\FunctionTok{=} \NormalTok{serializing' }\StringTok{"state.dat"} \FunctionTok{$} \NormalTok{mconcat [module1, module2, module3]}
\end{Highlighting}
\end{Shaded}

And now our \texttt{chatBot} will automatically resume itself on program
startup, and keep its state backed up on disk at \texttt{state.dat}. We
get this for free, without doing anything extra in the composition of
our modules.

Note that in practice, with a bot you are actively developing, this
might not be the best idea. \texttt{serializing\textquotesingle{}}
\emph{analyzes} your \texttt{Auto}s to determine a serialization and
reloading strategy, and applies that to do its job. However, if you, for
example, add a new module to your chat bot\ldots{}the serialization
strategy will change, and your new bot won't be able to resume old save
files.

One solution at this point is just to serialize individual modules that
you do not see yourself changing\ldots{}or even just serializing parts
of the modules you don't see yourself changing. Then you can change each
portion separately and not worry about migrtion issues.

\begin{Shaded}
\begin{Highlighting}[]
\OtherTok{chatBot ::} \DataTypeTok{MonadIO} \NormalTok{m }\OtherTok{=>} \DataTypeTok{ChatBot} \NormalTok{m}
\NormalTok{chatBot }\FunctionTok{=} \NormalTok{mconcat [ serializing' }\StringTok{"m1.dat"} \NormalTok{module1}
                  \NormalTok{, module2}
                  \NormalTok{, serializing' }\StringTok{"m3.dat"} \NormalTok{module3}
                  \NormalTok{]}
\end{Highlighting}
\end{Shaded}

We're not all-or-nothing now here, either! So, \texttt{module1} gets
serialized and auto-resumed from \texttt{m1.dat}, \texttt{module2} is
not serialized at all, and \texttt{module3} now gets serialized and
auto-resumed from \texttt{m3.dat}.

\section{IRC Backend (the ugly part)}\label{irc-backend-the-ugly-part}

Before we get started on our actual modules, let's just write out the
backend/interface between our \texttt{ChatBot} and irc to get it out of
the way. This will vary based on what library you use; I'm going to use
the
\href{http://hackage.haskell.org/package/simpleirc}{simpleirc-0.3.0},
but feel free to use any interface/library you want.

\begin{Shaded}
\begin{Highlighting}[]
\CommentTok{-- source: https://github.com/mstksg/blog/tree/develop/code-samples/auto/chatbot.hs#L25-199}
\OtherTok{withIrcConf ::} \DataTypeTok{IrcConfig} \OtherTok{->} \DataTypeTok{ChatBot} \DataTypeTok{IO} \OtherTok{->} \DataTypeTok{IO} \NormalTok{()}
\NormalTok{withIrcConf ircconf chatbot }\FunctionTok{=} \KeywordTok{do}

    \CommentTok{-- chan to receive `InMessage`s}
    \NormalTok{inputChan }\OtherTok{<- newChan ::} \DataTypeTok{IO} \NormalTok{(}\DataTypeTok{Chan} \DataTypeTok{InMessage}\NormalTok{)}

    \CommentTok{-- configuring IRC}
    \KeywordTok{let} \NormalTok{events   }\FunctionTok{=} \NormalTok{cEvents ircconf }\FunctionTok{++} \NormalTok{[ }\DataTypeTok{Privmsg} \NormalTok{(onMessage inputChan) ]}
        \NormalTok{ircconf' }\FunctionTok{=} \NormalTok{ircconf \{ cEvents }\FunctionTok{=} \NormalTok{events \}}

    \CommentTok{-- connect; simplified for demonstration purposes}
    \DataTypeTok{Right} \NormalTok{server }\OtherTok{<-} \NormalTok{connect ircconf' }\DataTypeTok{True} \DataTypeTok{True}

    \CommentTok{-- run `chatbot` on `inputChan`}
    \NormalTok{void }\FunctionTok{.} \NormalTok{forkIO }\FunctionTok{.} \NormalTok{void }\FunctionTok{$}
        \NormalTok{runOnChanM id (processOutput server) inputChan chatbot}

  \KeywordTok{where}
    \CommentTok{-- what to do when `chatBot` outputs}
\OtherTok{    processOutput ::} \DataTypeTok{MIrc} \OtherTok{->} \DataTypeTok{OutMessages} \OtherTok{->} \DataTypeTok{IO} \DataTypeTok{Bool}
    \NormalTok{processOutput server (}\DataTypeTok{OutMessages} \NormalTok{outs) }\FunctionTok{=} \KeywordTok{do}
      \NormalTok{print outs}
      \NormalTok{_ }\OtherTok{<-} \NormalTok{flip M.traverseWithKey outs }\FunctionTok{$} \NormalTok{\textbackslash{}channel messages }\OtherTok{->} \KeywordTok{do}
        \KeywordTok{let} \NormalTok{channel' }\FunctionTok{=} \NormalTok{encodeUtf8 }\FunctionTok{.} \NormalTok{pack }\FunctionTok{$} \NormalTok{channel}
        \NormalTok{forM_ messages }\FunctionTok{$} \NormalTok{\textbackslash{}message }\OtherTok{->} \KeywordTok{do}
          \KeywordTok{let} \NormalTok{message' }\FunctionTok{=} \NormalTok{encodeUtf8 }\FunctionTok{.} \NormalTok{pack }\FunctionTok{$} \NormalTok{message}
          \NormalTok{sendMsg server channel' message'}
      \NormalTok{return }\DataTypeTok{True}       \CommentTok{-- "yes, continue on"}

    \CommentTok{-- what to do when you get a new message}
\OtherTok{    onMessage ::} \DataTypeTok{Chan} \DataTypeTok{InMessage} \OtherTok{->} \DataTypeTok{EventFunc}
    \NormalTok{onMessage inputChan }\FunctionTok{=} \NormalTok{\textbackslash{}_ message }\OtherTok{->} \KeywordTok{do}
      \KeywordTok{case} \NormalTok{(mNick message, mOrigin message) }\KeywordTok{of}
        \NormalTok{(}\DataTypeTok{Just} \NormalTok{nick, }\DataTypeTok{Just} \NormalTok{src) }\OtherTok{->} \KeywordTok{do}
          \NormalTok{time }\OtherTok{<-} \NormalTok{getCurrentTime}
          \NormalTok{writeChan inputChan }\FunctionTok{$} \DataTypeTok{InMessage} \NormalTok{(unpack (decodeUtf8 nick))}
                                          \NormalTok{(unpack (decodeUtf8 (mMsg message)))}
                                          \NormalTok{(unpack (decodeUtf8 src))}
                                          \NormalTok{time}

\OtherTok{conf ::} \DataTypeTok{IrcConfig}
\NormalTok{conf }\FunctionTok{=} \NormalTok{(mkDefaultConfig }\StringTok{"irc.freenode.org"} \StringTok{"testautobot"}\NormalTok{) \{ cChannels }\FunctionTok{=} \NormalTok{[}\StringTok{"#jlebot-test"}\NormalTok{] \}}

\OtherTok{main ::} \DataTypeTok{IO} \NormalTok{()}
\NormalTok{main }\FunctionTok{=} \KeywordTok{do}
    \NormalTok{withIrcConf conf chatBot}
    \NormalTok{forever (threadDelay }\DecValTok{1000000000}\NormalTok{)}
\end{Highlighting}
\end{Shaded}

That should be it\ldots{}don't worry if you don't understand all of it,
most of it is just implementation details from \texttt{simpleirc}. The
overall loop is \texttt{runOnChanM} waits on a separate thread for
\texttt{inputChan}\ldots{}when it gets input, it runs it through
\texttt{ChatBot} and sends the outputs through \emph{simpleirc}'s
interface. Meanwhile, \texttt{onMessage} is triggered whenever
\emph{simpleirc} receives a message, where it prepares an
\texttt{InMessage} and drops it off at \texttt{inputChan}.

\begin{Shaded}
\begin{Highlighting}[]
\OtherTok{runOnChanM ::} \DataTypeTok{Monad} \NormalTok{m}
           \OtherTok{=>} \NormalTok{(forall c}\FunctionTok{.} \NormalTok{m c }\OtherTok{->} \DataTypeTok{IO} \NormalTok{c)   }\CommentTok{-- convert `m` to `IO`}
           \OtherTok{->} \NormalTok{(b }\OtherTok{->} \DataTypeTok{IO} \DataTypeTok{Bool}\NormalTok{)            }\CommentTok{-- handle output}
           \OtherTok{->} \DataTypeTok{Chan} \NormalTok{a                    }\CommentTok{-- chan to await input on}
           \OtherTok{->} \DataTypeTok{Auto} \NormalTok{m a b                }\CommentTok{-- `Auto` to run}
           \OtherTok{->} \DataTypeTok{IO} \NormalTok{(}\DataTypeTok{Auto} \NormalTok{m a b)}
\end{Highlighting}
\end{Shaded}

\texttt{runOnChanM} runs any \texttt{Auto\ m\ a\ b}, as long as there's
a way to convert it to \texttt{Auto\ IO\ a\ b} (we can use a
\texttt{ChatBot\ IO}, so we just put \texttt{id} there). You give it a
``handler'' \texttt{b\ -\textgreater{}\ IO\ Bool} that it run whenever
it outputs; if the handler returns \texttt{False}, then the whole thing
stops. You give it the \texttt{Chan\ a} to await for input \texttt{a}s
on, and it takes care of the rest. It blocks until the handler returns
\texttt{False}, where it'll return the ``updated''
\texttt{Auto\ m\ a\ b} with updated state after running through all of
those inputs.

Phew. With that out of the way, let's get right on to the fun part ---
building our chat bot modules.

\section{The Modules}\label{the-modules}

\subsection{seenBot}\label{seenbot}

What's a common module? Well, we can write a module that keeps track of
the last time any user was ``seen'' (sent a message), and then respond
when there is a query.

There are two components here\ldots{}the part that keeps track of the
last seen time, and the part that responds to queries.

Keeping track of our last seen time sounds like a job that takes in a
stream of \texttt{(Nick,\ UTCTime)} pairs and outputs a stream of
\texttt{Map\ Nick\ UTCTime}, where we could look up the last seen time
for a nick by looking up the nick in the map.

Logically, this is pretty straightforward, and anything other than
\texttt{accum} (which is like \texttt{foldl\textquotesingle{}}) would
really be a bit overkill; every input would just update the output map.

\begin{Shaded}
\begin{Highlighting}[]
\CommentTok{-- source: https://github.com/mstksg/blog/tree/develop/code-samples/auto/chatbot.hs#L80-86}
\OtherTok{    trackSeens ::} \DataTypeTok{Monad} \NormalTok{m }\OtherTok{=>} \DataTypeTok{Auto} \NormalTok{m (}\DataTypeTok{Nick}\NormalTok{, }\DataTypeTok{UTCTime}\NormalTok{) (}\DataTypeTok{Map} \DataTypeTok{Nick} \DataTypeTok{UTCTime}\NormalTok{)}
    \NormalTok{trackSeens }\FunctionTok{=} \NormalTok{accum (\textbackslash{}mp (nick, time) }\OtherTok{->} \NormalTok{M.insert nick time mp) M.empty}
\OtherTok{    queryBlips ::} \DataTypeTok{Auto} \NormalTok{m }\DataTypeTok{Message} \NormalTok{(}\DataTypeTok{Blip} \DataTypeTok{Nick}\NormalTok{)}
    \NormalTok{queryBlips }\FunctionTok{=} \NormalTok{emitJusts (getRequest }\FunctionTok{.} \NormalTok{words)}
      \KeywordTok{where}
        \NormalTok{getRequest (}\StringTok{"@seen"}\FunctionTok{:}\NormalTok{nick}\FunctionTok{:}\NormalTok{_) }\FunctionTok{=} \DataTypeTok{Just} \NormalTok{nick}
        \NormalTok{getRequest _                }\FunctionTok{=} \DataTypeTok{Nothing}



\OtherTok{trackSeens ::} \DataTypeTok{Monad} \NormalTok{m }\OtherTok{=>} \DataTypeTok{Auto} \NormalTok{m (}\DataTypeTok{Nick}\NormalTok{, }\DataTypeTok{UTCTime}\NormalTok{) (}\DataTypeTok{Map} \DataTypeTok{Nick} \DataTypeTok{UTCTime}\NormalTok{)}
\NormalTok{trackSeens }\FunctionTok{=} \NormalTok{accum (\textbackslash{}mp (nick, time) }\OtherTok{->} \NormalTok{M.insert nick time mp) M.empty}
\end{Highlighting}
\end{Shaded}

\texttt{accum} takes the same thing that \texttt{foldl} takes:

\begin{Shaded}
\begin{Highlighting}[]
\NormalTok{foldl}\OtherTok{ ::}            \NormalTok{(b }\OtherTok{->} \NormalTok{a }\OtherTok{->} \NormalTok{b) }\OtherTok{->} \NormalTok{b }\OtherTok{->} \NormalTok{[a] }\OtherTok{->} \NormalTok{b}
\OtherTok{accum ::} \DataTypeTok{Monad} \NormalTok{m }\OtherTok{=>} \NormalTok{(b }\OtherTok{->} \NormalTok{a }\OtherTok{->} \NormalTok{b) }\OtherTok{->} \NormalTok{b }\OtherTok{->} \DataTypeTok{Auto} \NormalTok{m a b}
\end{Highlighting}
\end{Shaded}

So it basically ``folds up'' the entire history of inputs, with a
starting value. Every time an input comes, the output is the new folded
history of inputs. You can sort of think of it as it applying the
function to any incoming values to an internal accumulator and updating
it at every step.

The next component is just to respond to requests. We want to do
something on some ``triggering'' input. Every once in a while, some
input will come that will ``trigger'' some special response. This is a
sign that we can use \emph{blip streams}.

\begin{Shaded}
\begin{Highlighting}[]
\CommentTok{-- source: https://github.com/mstksg/blog/tree/develop/code-samples/auto/chatbot.hs#L82-86}
\OtherTok{    queryBlips ::} \DataTypeTok{Auto} \NormalTok{m }\DataTypeTok{Message} \NormalTok{(}\DataTypeTok{Blip} \DataTypeTok{Nick}\NormalTok{)}
    \NormalTok{queryBlips }\FunctionTok{=} \NormalTok{emitJusts (getRequest }\FunctionTok{.} \NormalTok{words)}
      \KeywordTok{where}
        \NormalTok{getRequest (}\StringTok{"@seen"}\FunctionTok{:}\NormalTok{nick}\FunctionTok{:}\NormalTok{_) }\FunctionTok{=} \DataTypeTok{Just} \NormalTok{nick}
        \NormalTok{getRequest _                }\FunctionTok{=} \DataTypeTok{Nothing}



\OtherTok{queryBlips ::} \DataTypeTok{Auto} \NormalTok{m }\DataTypeTok{Message} \NormalTok{(}\DataTypeTok{Blip} \DataTypeTok{Nick}\NormalTok{)}
\NormalTok{queryBlips }\FunctionTok{=} \NormalTok{emitJusts (getRequest }\FunctionTok{.} \NormalTok{words)}
  \KeywordTok{where}
    \NormalTok{getRequest (}\StringTok{"@seen"}\FunctionTok{:}\NormalTok{nick}\FunctionTok{:}\NormalTok{_) }\FunctionTok{=} \DataTypeTok{Just} \NormalTok{nick}
    \NormalTok{getRequest _                }\FunctionTok{=} \DataTypeTok{Nothing}
\end{Highlighting}
\end{Shaded}

\texttt{queryBlips} takes an input stream of strings and turns it into
an output \emph{blip stream} that emits with a \texttt{Nick} whenever
the input stream contains a request in the form of
\texttt{"@seen\ {[}nick{]}"}.

With these simple blocks, we can build our \texttt{seenBot}:

\begin{Shaded}
\begin{Highlighting}[]
\CommentTok{-- seenBot :: Monad m => Auto m InMessage (Blip [Message])}
\CommentTok{-- source: https://github.com/mstksg/blog/tree/develop/code-samples/auto/chatbot.hs#L67-86}
\OtherTok{seenBot ::} \DataTypeTok{Monad} \NormalTok{m }\OtherTok{=>} \DataTypeTok{RoomBot} \NormalTok{m}
\NormalTok{seenBot }\FunctionTok{=} \NormalTok{proc (}\DataTypeTok{InMessage} \NormalTok{nick msg _ time) }\OtherTok{->} \KeywordTok{do}
    \NormalTok{seens  }\OtherTok{<-} \NormalTok{trackSeens }\FunctionTok{-<} \NormalTok{(nick, time)}

    \NormalTok{queryB }\OtherTok{<-} \NormalTok{queryBlips }\FunctionTok{-<} \NormalTok{msg}

    \KeywordTok{let}\OtherTok{ respond ::} \DataTypeTok{Nick} \OtherTok{->} \NormalTok{[}\DataTypeTok{Message}\NormalTok{]}
        \NormalTok{respond qry }\FunctionTok{=} \KeywordTok{case} \NormalTok{M.lookup qry seens }\KeywordTok{of}
                        \DataTypeTok{Just} \NormalTok{t  }\OtherTok{->} \NormalTok{[qry }\FunctionTok{++} \StringTok{" last seen at "} \FunctionTok{++} \NormalTok{show t }\FunctionTok{++} \StringTok{"."}\NormalTok{]}
                        \DataTypeTok{Nothing} \OtherTok{->} \NormalTok{[}\StringTok{"No record of "} \FunctionTok{++} \NormalTok{qry }\FunctionTok{++} \StringTok{"."}\NormalTok{]}

    \NormalTok{id }\FunctionTok{-<} \NormalTok{respond }\FunctionTok{<$>} \NormalTok{queryB}
  \KeywordTok{where}
\OtherTok{    trackSeens ::} \DataTypeTok{Monad} \NormalTok{m }\OtherTok{=>} \DataTypeTok{Auto} \NormalTok{m (}\DataTypeTok{Nick}\NormalTok{, }\DataTypeTok{UTCTime}\NormalTok{) (}\DataTypeTok{Map} \DataTypeTok{Nick} \DataTypeTok{UTCTime}\NormalTok{)}
    \NormalTok{trackSeens }\FunctionTok{=} \NormalTok{accum (\textbackslash{}mp (nick, time) }\OtherTok{->} \NormalTok{M.insert nick time mp) M.empty}
\OtherTok{    queryBlips ::} \DataTypeTok{Auto} \NormalTok{m }\DataTypeTok{Message} \NormalTok{(}\DataTypeTok{Blip} \DataTypeTok{Nick}\NormalTok{)}
    \NormalTok{queryBlips }\FunctionTok{=} \NormalTok{emitJusts (getRequest }\FunctionTok{.} \NormalTok{words)}
      \KeywordTok{where}
        \NormalTok{getRequest (}\StringTok{"@seen"}\FunctionTok{:}\NormalTok{nick}\FunctionTok{:}\NormalTok{_) }\FunctionTok{=} \DataTypeTok{Just} \NormalTok{nick}
        \NormalTok{getRequest _                }\FunctionTok{=} \DataTypeTok{Nothing}
\end{Highlighting}
\end{Shaded}

Here we define \texttt{respond} as a function that takes a \texttt{Nick}
and returns the output \texttt{{[}Message{]}}. We could have also
defined it outside as a helper function
\texttt{respond\ ::\ Map\ Nick\ UTCTime\ -\textgreater{}\ Nick\ -\textgreater{}\ {[}Message{]}}\ldots{}but
\texttt{seens} is already in scope, so we might as well just do it
there.

For our output, we use the \texttt{Functor} instance of blip streams.
\texttt{respond\ \textless{}\$\textgreater{}\ queryB} is a blip stream
that emits whenever \texttt{queryB} emits (so, whenever there is a query
input), but replaces the emitted value with the result of the function
on the value. So whenever \texttt{queryB} emits, this whole thing emits
with \texttt{respond} applied to whatever \texttt{Nick} was emitted ---
in this case, our \texttt{{[}Message{]}}.

Short, sweet, simple. In fact, \texttt{trackSeens} and
\texttt{queryBlips} are small enough that their definition could really
have been inlined. Breaking them down just allowed us to look at them
individually for this tutorial.

So that's it for that; also, if we wanted \texttt{seenBot} to serialize
and persist across sessions, all we have to do is use:

\begin{Shaded}
\begin{Highlighting}[]
\NormalTok{serializing' }\StringTok{"seenbot.dat"}\OtherTok{ seenBot ::} \DataTypeTok{MonadIO} \NormalTok{m }\OtherTok{=>} \DataTypeTok{RoomBot} \NormalTok{m}
\end{Highlighting}
\end{Shaded}

Neat, right?

If we forsee ourselves adding more features to \texttt{seenBot}, we can
future-proof our \texttt{seenBot} for now by only serializing
\texttt{trackSeens}, meaning replacing that line with:

\begin{Shaded}
\begin{Highlighting}[]
    \NormalTok{seens }\OtherTok{<-} \NormalTok{serializing' }\StringTok{"seen.dat"} \NormalTok{trackSeens }\FunctionTok{-<} \NormalTok{(nick, time)}
\end{Highlighting}
\end{Shaded}

Remember, \texttt{serializing\textquotesingle{}\ fp} acts as a sort of
``identity'', so you can drop it in anywhere and you'd expect it to
behave the same.

\subsection{repBot}\label{repbot}

Another common bot is a ``reputation bot'', which allows users to
increment or decrement another user's reputation scores, and look up a
user's total score.

Again there are two components --- keeping track of the scores of all of
the users, and responding to requests.

This time though, our ``score updates'' only happen every once in a
while, triggered by certain words in the message. Again, this pattern
calls for a blip stream:

\begin{Shaded}
\begin{Highlighting}[]
\CommentTok{-- source: https://github.com/mstksg/blog/tree/develop/code-samples/auto/chatbot.hs#L103-118}
\OtherTok{    updateBlips ::} \DataTypeTok{Auto} \NormalTok{m (}\DataTypeTok{Nick}\NormalTok{, }\DataTypeTok{Message}\NormalTok{) (}\DataTypeTok{Blip} \NormalTok{(}\DataTypeTok{Nick}\NormalTok{, }\DataTypeTok{Int}\NormalTok{))}
    \NormalTok{updateBlips }\FunctionTok{=} \NormalTok{emitJusts getUpdateCommand}
      \KeywordTok{where}
        \CommentTok{-- updater is the person triggering the update blip}
        \NormalTok{getUpdateCommand (updater, msg) }\FunctionTok{=}
          \KeywordTok{case} \NormalTok{words msg }\KeywordTok{of}
            \StringTok{"@addRep"}\FunctionTok{:}\NormalTok{nick}\FunctionTok{:}\NormalTok{_ }\FunctionTok{|} \NormalTok{nick }\FunctionTok{/=} \NormalTok{updater }\OtherTok{->} \DataTypeTok{Just} \NormalTok{(nick, }\DecValTok{1}\NormalTok{)}
            \StringTok{"@subRep"}\FunctionTok{:}\NormalTok{nick}\FunctionTok{:}\NormalTok{_                   }\OtherTok{->} \DataTypeTok{Just} \NormalTok{(nick, }\FunctionTok{-}\DecValTok{1}\NormalTok{)}
            \NormalTok{_                                  }\OtherTok{->} \DataTypeTok{Nothing}
\OtherTok{    trackReps ::} \DataTypeTok{Monad} \NormalTok{m }\OtherTok{=>} \DataTypeTok{Auto} \NormalTok{m (}\DataTypeTok{Blip} \NormalTok{(}\DataTypeTok{Nick}\NormalTok{, }\DataTypeTok{Int}\NormalTok{)) (}\DataTypeTok{Map} \DataTypeTok{Nick} \DataTypeTok{Int}\NormalTok{)}
    \NormalTok{trackReps }\FunctionTok{=} \NormalTok{scanB (\textbackslash{}mp (nick, change) }\OtherTok{->} \NormalTok{M.insertWith (}\FunctionTok{+}\NormalTok{) nick change mp) M.empty}
\OtherTok{    queryBlips ::} \DataTypeTok{Auto} \NormalTok{m }\DataTypeTok{Message} \NormalTok{(}\DataTypeTok{Blip} \DataTypeTok{Nick}\NormalTok{)}
    \NormalTok{queryBlips }\FunctionTok{=} \NormalTok{emitJusts (getRequest }\FunctionTok{.} \NormalTok{words)}
      \KeywordTok{where}
        \NormalTok{getRequest (}\StringTok{"@rep"}\FunctionTok{:}\NormalTok{nick}\FunctionTok{:}\NormalTok{_) }\FunctionTok{=} \DataTypeTok{Just} \NormalTok{nick}
        \NormalTok{getRequest _                }\FunctionTok{=} \DataTypeTok{Nothing}



\OtherTok{updateBlips ::} \DataTypeTok{Auto} \NormalTok{m (}\DataTypeTok{Nick}\NormalTok{, }\DataTypeTok{Message}\NormalTok{) (}\DataTypeTok{Blip} \NormalTok{(}\DataTypeTok{Nick}\NormalTok{, }\DataTypeTok{Int}\NormalTok{))}
\NormalTok{updateBlips }\FunctionTok{=} \NormalTok{emitJusts getUpdateCommand}
  \KeywordTok{where}
    \CommentTok{-- updater is the person triggering the update blip}
    \NormalTok{getUpdateCommand (updater, msg) }\FunctionTok{=}
      \KeywordTok{case} \NormalTok{words msg }\KeywordTok{of}
        \StringTok{"@addRep"}\FunctionTok{:}\NormalTok{nick}\FunctionTok{:}\NormalTok{_ }\FunctionTok{|} \NormalTok{nick }\FunctionTok{/=} \NormalTok{updater }\OtherTok{->} \DataTypeTok{Just} \NormalTok{(nick, }\DecValTok{1}\NormalTok{)}
        \StringTok{"@subRep"}\FunctionTok{:}\NormalTok{nick}\FunctionTok{:}\NormalTok{_                   }\OtherTok{->} \DataTypeTok{Just} \NormalTok{(nick, }\FunctionTok{-}\DecValTok{1}\NormalTok{)}
        \NormalTok{_                                  }\OtherTok{->} \DataTypeTok{Nothing}
\end{Highlighting}
\end{Shaded}

\texttt{updateBlips} takes in a \texttt{(Nick,\ Message)} blip, with the
person who is sending the message and their message, and emit with a
\texttt{(Nick,\ Int)} whenever the message is a command. The emitted
\texttt{(Nick,\ Int)} has the person to adjust, and the amount to adjust
by. Note that we ignore commands where the person is trying to increase
their own reputation because that's just lame.

We probably want to keep track of the scores as a
\texttt{Map\ Nick\ Int}, so we can do that with something like
\texttt{accum} again. However, \texttt{accum} takes a stream of normal
values, but we have a \emph{blip stream}, so we can use \texttt{scanB}
instead. \texttt{scanB} is pretty much the same thing, but it collapses
a blip stream into a value stream by holding the ``current result'' of
the fold.\footnote{\texttt{scanB\ f\ x0\ ::\ Auto\ m\ (Blip\ a)\ b}, but
  there's also
  \texttt{accumB\ f\ x0\ ::\ Auto\ m\ a\ (Blip\ a)\ (Blip\ b)}, which
  emits whenever the input emits only.}

\begin{Shaded}
\begin{Highlighting}[]
\CommentTok{-- source: https://github.com/mstksg/blog/tree/develop/code-samples/auto/chatbot.hs#L112-118}
\OtherTok{    trackReps ::} \DataTypeTok{Monad} \NormalTok{m }\OtherTok{=>} \DataTypeTok{Auto} \NormalTok{m (}\DataTypeTok{Blip} \NormalTok{(}\DataTypeTok{Nick}\NormalTok{, }\DataTypeTok{Int}\NormalTok{)) (}\DataTypeTok{Map} \DataTypeTok{Nick} \DataTypeTok{Int}\NormalTok{)}
    \NormalTok{trackReps }\FunctionTok{=} \NormalTok{scanB (\textbackslash{}mp (nick, change) }\OtherTok{->} \NormalTok{M.insertWith (}\FunctionTok{+}\NormalTok{) nick change mp) M.empty}
\OtherTok{    queryBlips ::} \DataTypeTok{Auto} \NormalTok{m }\DataTypeTok{Message} \NormalTok{(}\DataTypeTok{Blip} \DataTypeTok{Nick}\NormalTok{)}
    \NormalTok{queryBlips }\FunctionTok{=} \NormalTok{emitJusts (getRequest }\FunctionTok{.} \NormalTok{words)}
      \KeywordTok{where}
        \NormalTok{getRequest (}\StringTok{"@rep"}\FunctionTok{:}\NormalTok{nick}\FunctionTok{:}\NormalTok{_) }\FunctionTok{=} \DataTypeTok{Just} \NormalTok{nick}
        \NormalTok{getRequest _                }\FunctionTok{=} \DataTypeTok{Nothing}



\OtherTok{trackReps ::} \DataTypeTok{Monad} \NormalTok{m }\OtherTok{=>} \DataTypeTok{Auto} \NormalTok{m (}\DataTypeTok{Blip} \NormalTok{(}\DataTypeTok{Nick}\NormalTok{, }\DataTypeTok{Int}\NormalTok{)) (}\DataTypeTok{Map} \DataTypeTok{Nick} \DataTypeTok{Int}\NormalTok{)}
\NormalTok{trackReps }\FunctionTok{=} \NormalTok{scanB (\textbackslash{}mp (nick, change) }\OtherTok{->} \NormalTok{M.insertWith (}\FunctionTok{+}\NormalTok{) nick change mp) M.empty}
\end{Highlighting}
\end{Shaded}

And finally, the ``response'' portion --- we want to be able to respond
to commands and look up the result. We basically had this identical
pattern for \texttt{seenBot}:

\begin{Shaded}
\begin{Highlighting}[]
\CommentTok{-- source: https://github.com/mstksg/blog/tree/develop/code-samples/auto/chatbot.hs#L82-86}
\OtherTok{    queryBlips ::} \DataTypeTok{Auto} \NormalTok{m }\DataTypeTok{Message} \NormalTok{(}\DataTypeTok{Blip} \DataTypeTok{Nick}\NormalTok{)}
    \NormalTok{queryBlips }\FunctionTok{=} \NormalTok{emitJusts (getRequest }\FunctionTok{.} \NormalTok{words)}
      \KeywordTok{where}
        \NormalTok{getRequest (}\StringTok{"@seen"}\FunctionTok{:}\NormalTok{nick}\FunctionTok{:}\NormalTok{_) }\FunctionTok{=} \DataTypeTok{Just} \NormalTok{nick}
        \NormalTok{getRequest _                }\FunctionTok{=} \DataTypeTok{Nothing}



\OtherTok{queryBlips ::} \DataTypeTok{Auto} \NormalTok{m }\DataTypeTok{Message} \NormalTok{(}\DataTypeTok{Blip} \DataTypeTok{Nick}\NormalTok{)}
\NormalTok{queryBlips }\FunctionTok{=} \NormalTok{emitJusts (getRequest }\FunctionTok{.} \NormalTok{words)}
  \KeywordTok{where}
    \NormalTok{getRequest (}\StringTok{"@rep"}\FunctionTok{:}\NormalTok{nick}\FunctionTok{:}\NormalTok{_) }\FunctionTok{=} \DataTypeTok{Just} \NormalTok{nick}
    \NormalTok{getRequest _                }\FunctionTok{=} \DataTypeTok{Nothing}
\end{Highlighting}
\end{Shaded}

And\ldots{}now we can wrap it all together with a nice proc block:

\begin{Shaded}
\begin{Highlighting}[]
\CommentTok{-- repBot :: Monad m => Auto m InMessage (Blip [Message])}
\CommentTok{-- source: https://github.com/mstksg/blog/tree/develop/code-samples/auto/chatbot.hs#L88-118}
\OtherTok{repBot ::} \DataTypeTok{Monad} \NormalTok{m }\OtherTok{=>} \DataTypeTok{RoomBot} \NormalTok{m}
\NormalTok{repBot }\FunctionTok{=} \NormalTok{proc (}\DataTypeTok{InMessage} \NormalTok{nick msg _ _) }\OtherTok{->} \KeywordTok{do}
    \NormalTok{updateB }\OtherTok{<-} \NormalTok{updateBlips }\FunctionTok{-<} \NormalTok{(nick, msg)}

    \NormalTok{reps    }\OtherTok{<-} \NormalTok{trackReps   }\FunctionTok{-<} \NormalTok{updateB}

    \NormalTok{queryB  }\OtherTok{<-} \NormalTok{queryBlips  }\FunctionTok{-<} \NormalTok{msg}

    \KeywordTok{let}\OtherTok{ lookupRep ::} \DataTypeTok{Nick} \OtherTok{->} \NormalTok{[}\DataTypeTok{Message}\NormalTok{]}
        \NormalTok{lookupRep nick }\FunctionTok{=} \NormalTok{[nick }\FunctionTok{++} \StringTok{" has a reputation of "} \FunctionTok{++} \NormalTok{show rep }\FunctionTok{++} \StringTok{"."}\NormalTok{]}
          \KeywordTok{where}
            \NormalTok{rep }\FunctionTok{=} \NormalTok{M.findWithDefault }\DecValTok{0} \NormalTok{nick reps}

    \NormalTok{id }\FunctionTok{-<} \NormalTok{lookupRep }\FunctionTok{<$>} \NormalTok{queryB}
  \KeywordTok{where}
\OtherTok{    updateBlips ::} \DataTypeTok{Auto} \NormalTok{m (}\DataTypeTok{Nick}\NormalTok{, }\DataTypeTok{Message}\NormalTok{) (}\DataTypeTok{Blip} \NormalTok{(}\DataTypeTok{Nick}\NormalTok{, }\DataTypeTok{Int}\NormalTok{))}
    \NormalTok{updateBlips }\FunctionTok{=} \NormalTok{emitJusts getUpdateCommand}
      \KeywordTok{where}
        \CommentTok{-- updater is the person triggering the update blip}
        \NormalTok{getUpdateCommand (updater, msg) }\FunctionTok{=}
          \KeywordTok{case} \NormalTok{words msg }\KeywordTok{of}
            \StringTok{"@addRep"}\FunctionTok{:}\NormalTok{nick}\FunctionTok{:}\NormalTok{_ }\FunctionTok{|} \NormalTok{nick }\FunctionTok{/=} \NormalTok{updater }\OtherTok{->} \DataTypeTok{Just} \NormalTok{(nick, }\DecValTok{1}\NormalTok{)}
            \StringTok{"@subRep"}\FunctionTok{:}\NormalTok{nick}\FunctionTok{:}\NormalTok{_                   }\OtherTok{->} \DataTypeTok{Just} \NormalTok{(nick, }\FunctionTok{-}\DecValTok{1}\NormalTok{)}
            \NormalTok{_                                  }\OtherTok{->} \DataTypeTok{Nothing}
\OtherTok{    trackReps ::} \DataTypeTok{Monad} \NormalTok{m }\OtherTok{=>} \DataTypeTok{Auto} \NormalTok{m (}\DataTypeTok{Blip} \NormalTok{(}\DataTypeTok{Nick}\NormalTok{, }\DataTypeTok{Int}\NormalTok{)) (}\DataTypeTok{Map} \DataTypeTok{Nick} \DataTypeTok{Int}\NormalTok{)}
    \NormalTok{trackReps }\FunctionTok{=} \NormalTok{scanB (\textbackslash{}mp (nick, change) }\OtherTok{->} \NormalTok{M.insertWith (}\FunctionTok{+}\NormalTok{) nick change mp) M.empty}
\OtherTok{    queryBlips ::} \DataTypeTok{Auto} \NormalTok{m }\DataTypeTok{Message} \NormalTok{(}\DataTypeTok{Blip} \DataTypeTok{Nick}\NormalTok{)}
    \NormalTok{queryBlips }\FunctionTok{=} \NormalTok{emitJusts (getRequest }\FunctionTok{.} \NormalTok{words)}
      \KeywordTok{where}
        \NormalTok{getRequest (}\StringTok{"@rep"}\FunctionTok{:}\NormalTok{nick}\FunctionTok{:}\NormalTok{_) }\FunctionTok{=} \DataTypeTok{Just} \NormalTok{nick}
        \NormalTok{getRequest _                }\FunctionTok{=} \DataTypeTok{Nothing}
\end{Highlighting}
\end{Shaded}

Again note that we take advantage of the \texttt{Functor} instance of
blip streams to create a new blip stream
(\texttt{lookupRep\ \textless{}\$\textgreater{}\ queryB}) that emits
whenever \texttt{queryB} emits, but replaces the value with
\texttt{lookupRep} applied to whatever \texttt{Nick} was in the query
blip. We also take advantage that \texttt{reps} is in scope and define
\texttt{lookupRep} right there in the block.

\subsection{announceBot}\label{announcebot}

Let's just go over one more module\ldots{}and I think you'll be able to
use your imagination to think of and implement your own from here.

Let's make an ``announceBot'', that listens for ``announcement''
messages from anyone (even in private messages) and broadcasts them to
all of the channels in the provided list. It rate-limits the
announcements, though, so that a user is only limited to three
announcements per day.

We can start with our typical ``blip stream that emits on a certain
command'' to start off everything:

\begin{Shaded}
\begin{Highlighting}[]
\CommentTok{-- source: https://github.com/mstksg/blog/tree/develop/code-samples/auto/chatbot.hs#L144-154}
\OtherTok{    announceBlips ::} \DataTypeTok{Monad} \NormalTok{m }\OtherTok{=>} \DataTypeTok{Auto} \NormalTok{m (}\DataTypeTok{Nick}\NormalTok{, }\DataTypeTok{Message}\NormalTok{) (}\DataTypeTok{Blip} \NormalTok{[}\DataTypeTok{Message}\NormalTok{])}
    \NormalTok{announceBlips }\FunctionTok{=} \NormalTok{emitJusts getAnnounces}
      \KeywordTok{where}
        \NormalTok{getAnnounces (nick, msg) }\FunctionTok{=}
          \KeywordTok{case} \NormalTok{words msg }\KeywordTok{of}
            \StringTok{"@ann"}\FunctionTok{:}\NormalTok{ann }\OtherTok{->} \DataTypeTok{Just} \NormalTok{[nick }\FunctionTok{++} \StringTok{" says \textbackslash{}""} \FunctionTok{++} \NormalTok{unwords ann }\FunctionTok{++} \StringTok{"\textbackslash{}"."}\NormalTok{]}
            \NormalTok{_          }\OtherTok{->} \DataTypeTok{Nothing}
\OtherTok{    newDayBlips ::} \DataTypeTok{Monad} \NormalTok{m }\OtherTok{=>} \DataTypeTok{Auto} \NormalTok{m }\DataTypeTok{Day} \NormalTok{(}\DataTypeTok{Blip} \DataTypeTok{Day}\NormalTok{)}
    \NormalTok{newDayBlips }\FunctionTok{=} \NormalTok{onChange}
\OtherTok{    trackAnns ::} \DataTypeTok{Monad} \NormalTok{m }\OtherTok{=>} \DataTypeTok{Auto} \NormalTok{m (}\DataTypeTok{Blip} \DataTypeTok{Nick}\NormalTok{) (}\DataTypeTok{Map} \DataTypeTok{Nick} \DataTypeTok{Int}\NormalTok{)}
    \NormalTok{trackAnns }\FunctionTok{=} \NormalTok{scanB (\textbackslash{}mp nick }\OtherTok{->} \NormalTok{M.insertWith (}\FunctionTok{+}\NormalTok{) nick }\DecValTok{1} \NormalTok{mp) M.empty}



\OtherTok{announceBlips ::} \DataTypeTok{Monad} \NormalTok{m }\OtherTok{=>} \DataTypeTok{Auto} \NormalTok{m (}\DataTypeTok{Nick}\NormalTok{, }\DataTypeTok{Message}\NormalTok{) (}\DataTypeTok{Blip} \NormalTok{[}\DataTypeTok{Message}\NormalTok{])}
\NormalTok{announceBlips }\FunctionTok{=} \NormalTok{emitJusts getAnnounces}
  \KeywordTok{where}
    \NormalTok{getAnnounces (nick, msg) }\FunctionTok{=}
      \KeywordTok{case} \NormalTok{words msg }\KeywordTok{of}
        \StringTok{"@ann"}\FunctionTok{:}\NormalTok{ann }\OtherTok{->} \DataTypeTok{Just} \NormalTok{[nick }\FunctionTok{++} \StringTok{" says \textbackslash{}""} \FunctionTok{++} \NormalTok{unwords ann }\FunctionTok{++} \StringTok{"\textbackslash{}"."}\NormalTok{]}
        \NormalTok{_          }\OtherTok{->} \DataTypeTok{Nothing}
\end{Highlighting}
\end{Shaded}

\texttt{announceBlips} takes in a nick-message pair and emits an
announcement \texttt{{[}Message{]}} whenever the incoming message is an
announcement command.

Next, we'd like to keep track of how many times a user has made an
announcement today. This is pretty much just \texttt{scanB} again like
with \texttt{repBot}:

\begin{Shaded}
\begin{Highlighting}[]
\CommentTok{-- source: https://github.com/mstksg/blog/tree/develop/code-samples/auto/chatbot.hs#L153-154}
\OtherTok{    trackAnns ::} \DataTypeTok{Monad} \NormalTok{m }\OtherTok{=>} \DataTypeTok{Auto} \NormalTok{m (}\DataTypeTok{Blip} \DataTypeTok{Nick}\NormalTok{) (}\DataTypeTok{Map} \DataTypeTok{Nick} \DataTypeTok{Int}\NormalTok{)}
    \NormalTok{trackAnns }\FunctionTok{=} \NormalTok{scanB (\textbackslash{}mp nick }\OtherTok{->} \NormalTok{M.insertWith (}\FunctionTok{+}\NormalTok{) nick }\DecValTok{1} \NormalTok{mp) M.empty}



\OtherTok{trackAnns ::} \DataTypeTok{Monad} \NormalTok{m }\OtherTok{=>} \DataTypeTok{Auto} \NormalTok{m (}\DataTypeTok{Blip} \DataTypeTok{Nick}\NormalTok{) (}\DataTypeTok{Map} \DataTypeTok{Nick} \DataTypeTok{Int}\NormalTok{)}
\NormalTok{trackAnns }\FunctionTok{=} \NormalTok{scanB (\textbackslash{}mp nick }\OtherTok{->} \NormalTok{M.insertWith (}\FunctionTok{+}\NormalTok{) nick }\DecValTok{1} \NormalTok{mp) M.empty}
\end{Highlighting}
\end{Shaded}

However, we'd like to be able to ``reset'' this map whenever a new day
arrives. For that, we can use \texttt{resetOn} from
\href{http://hackage.haskell.org/package/auto/docs/Control-Auto-Switch.html}{\texttt{Control.Auto.Switch}},
which takes an \texttt{Auto} and gives it a ``reset channel'' input blip
stream, that resets the whole thing whenever the blip stream emits:

\begin{Shaded}
\begin{Highlighting}[]
\OtherTok{resetOn ::} \DataTypeTok{Monad} \NormalTok{m }\OtherTok{=>} \DataTypeTok{Auto} \NormalTok{m a b }\OtherTok{->} \DataTypeTok{Auto} \NormalTok{m (a        , }\DataTypeTok{Blip} \NormalTok{c) b}

\NormalTok{resetOn}\OtherTok{ trackAnns ::} \DataTypeTok{Monad} \NormalTok{m }\OtherTok{=>}     \DataTypeTok{Auto} \NormalTok{m (}\DataTypeTok{Blip} \DataTypeTok{Nick}\NormalTok{, }\DataTypeTok{Blip} \NormalTok{c) (}\DataTypeTok{Map} \DataTypeTok{Nick} \DataTypeTok{Int}\NormalTok{)}
\end{Highlighting}
\end{Shaded}

(It doesn't care about the actual value emitted, so we can leave it as a
type variable \texttt{c} conceptually.)

Now the only thing we need is a blip stream that emits whenever there is
a new day. For that, we can use \texttt{onChange} from
\href{http://hackage.haskell.org/package/auto/docs/Control-Auto-Blip.html}{\texttt{Control.Auto.Blip}}:

\begin{Shaded}
\begin{Highlighting}[]
\CommentTok{-- source: https://github.com/mstksg/blog/tree/develop/code-samples/auto/chatbot.hs#L151-154}
\OtherTok{    newDayBlips ::} \DataTypeTok{Monad} \NormalTok{m }\OtherTok{=>} \DataTypeTok{Auto} \NormalTok{m }\DataTypeTok{Day} \NormalTok{(}\DataTypeTok{Blip} \DataTypeTok{Day}\NormalTok{)}
    \NormalTok{newDayBlips }\FunctionTok{=} \NormalTok{onChange}
\OtherTok{    trackAnns ::} \DataTypeTok{Monad} \NormalTok{m }\OtherTok{=>} \DataTypeTok{Auto} \NormalTok{m (}\DataTypeTok{Blip} \DataTypeTok{Nick}\NormalTok{) (}\DataTypeTok{Map} \DataTypeTok{Nick} \DataTypeTok{Int}\NormalTok{)}
    \NormalTok{trackAnns }\FunctionTok{=} \NormalTok{scanB (\textbackslash{}mp nick }\OtherTok{->} \NormalTok{M.insertWith (}\FunctionTok{+}\NormalTok{) nick }\DecValTok{1} \NormalTok{mp) M.empty}


\OtherTok{newDayBlips ::} \DataTypeTok{Monad} \NormalTok{m }\OtherTok{=>} \DataTypeTok{Auto} \NormalTok{m }\DataTypeTok{Day} \NormalTok{(}\DataTypeTok{Blip} \DataTypeTok{Day}\NormalTok{)}
\NormalTok{newDayBlips }\FunctionTok{=} \NormalTok{onChange}
\end{Highlighting}
\end{Shaded}

\texttt{newDayBlips} takes in a stream of \texttt{Day}s (from
\texttt{Data.Time}) that we get from the \texttt{InMessage} and outputs
a blip stream that emits whenever the day changes. It emits with the new
\texttt{Day}\ldots{}but we don't really care about the emitted value,
we're just using it to triger \texttt{resetOn\ trackAnns}.

Finally, let's wrap it all together!

Remember, \texttt{announceBot} is a full on \texttt{ChatBot\ m}, and not
a \texttt{RoomBot\ m} anymore, so it has to say where it wants to send
its messages.

\begin{Shaded}
\begin{Highlighting}[]
\CommentTok{-- announceBot :: Monad m => [Channel] -> Auto m InMessage OutMessages}
\CommentTok{-- source: https://github.com/mstksg/blog/tree/develop/code-samples/auto/chatbot.hs#L120-154}
\OtherTok{announceBot ::} \DataTypeTok{Monad} \NormalTok{m }\OtherTok{=>} \NormalTok{[}\DataTypeTok{Channel}\NormalTok{] }\OtherTok{->} \DataTypeTok{ChatBot} \NormalTok{m}
\NormalTok{announceBot chans }\FunctionTok{=} \NormalTok{proc (}\DataTypeTok{InMessage} \NormalTok{nick msg src time) }\OtherTok{->} \KeywordTok{do}
    \NormalTok{announceB }\OtherTok{<-} \NormalTok{announceBlips     }\FunctionTok{-<} \NormalTok{(nick, msg)}

    \NormalTok{newDayB   }\OtherTok{<-} \NormalTok{newDayBlips       }\FunctionTok{-<} \NormalTok{utctDay time}

    \NormalTok{annCounts }\OtherTok{<-} \NormalTok{resetOn trackAnns }\FunctionTok{-<} \NormalTok{(nick }\FunctionTok{<$} \NormalTok{announceB, newDayB)}

    \KeywordTok{let} \NormalTok{hasFlooded  }\FunctionTok{=} \NormalTok{M.findWithDefault }\DecValTok{0} \NormalTok{nick annCounts }\FunctionTok{>} \DecValTok{3}

\OtherTok{        targetChans ::} \NormalTok{[}\DataTypeTok{Channel}\NormalTok{]}
        \NormalTok{targetChans }\FunctionTok{|} \NormalTok{hasFlooded }\FunctionTok{=} \NormalTok{[src]}
                    \FunctionTok{|} \NormalTok{otherwise  }\FunctionTok{=} \NormalTok{chans}

\OtherTok{        outB        ::} \DataTypeTok{Blip} \NormalTok{[}\DataTypeTok{Message}\NormalTok{]}
        \NormalTok{outB        }\FunctionTok{|} \NormalTok{hasFlooded }\FunctionTok{=} \NormalTok{[nick }\FunctionTok{++} \StringTok{": No flooding!"}\NormalTok{] }\FunctionTok{<$} \NormalTok{announceB}
                    \FunctionTok{|} \NormalTok{otherwise  }\FunctionTok{=} \NormalTok{announceB}

\OtherTok{        outMsgsB    ::} \DataTypeTok{Blip} \DataTypeTok{OutMessages}
        \NormalTok{outMsgsB    }\FunctionTok{=} \NormalTok{(\textbackslash{}out }\OtherTok{->} \DataTypeTok{OutMessages} \NormalTok{(M.fromList (map (,out) targetChans)))}
                  \FunctionTok{<$>} \NormalTok{outB}

    \NormalTok{fromBlips mempty }\FunctionTok{-<} \NormalTok{outMsgsB}
  \KeywordTok{where}
\OtherTok{    announceBlips ::} \DataTypeTok{Monad} \NormalTok{m }\OtherTok{=>} \DataTypeTok{Auto} \NormalTok{m (}\DataTypeTok{Nick}\NormalTok{, }\DataTypeTok{Message}\NormalTok{) (}\DataTypeTok{Blip} \NormalTok{[}\DataTypeTok{Message}\NormalTok{])}
    \NormalTok{announceBlips }\FunctionTok{=} \NormalTok{emitJusts getAnnounces}
      \KeywordTok{where}
        \NormalTok{getAnnounces (nick, msg) }\FunctionTok{=}
          \KeywordTok{case} \NormalTok{words msg }\KeywordTok{of}
            \StringTok{"@ann"}\FunctionTok{:}\NormalTok{ann }\OtherTok{->} \DataTypeTok{Just} \NormalTok{[nick }\FunctionTok{++} \StringTok{" says \textbackslash{}""} \FunctionTok{++} \NormalTok{unwords ann }\FunctionTok{++} \StringTok{"\textbackslash{}"."}\NormalTok{]}
            \NormalTok{_          }\OtherTok{->} \DataTypeTok{Nothing}
\OtherTok{    newDayBlips ::} \DataTypeTok{Monad} \NormalTok{m }\OtherTok{=>} \DataTypeTok{Auto} \NormalTok{m }\DataTypeTok{Day} \NormalTok{(}\DataTypeTok{Blip} \DataTypeTok{Day}\NormalTok{)}
    \NormalTok{newDayBlips }\FunctionTok{=} \NormalTok{onChange}
\OtherTok{    trackAnns ::} \DataTypeTok{Monad} \NormalTok{m }\OtherTok{=>} \DataTypeTok{Auto} \NormalTok{m (}\DataTypeTok{Blip} \DataTypeTok{Nick}\NormalTok{) (}\DataTypeTok{Map} \DataTypeTok{Nick} \DataTypeTok{Int}\NormalTok{)}
    \NormalTok{trackAnns }\FunctionTok{=} \NormalTok{scanB (\textbackslash{}mp nick }\OtherTok{->} \NormalTok{M.insertWith (}\FunctionTok{+}\NormalTok{) nick }\DecValTok{1} \NormalTok{mp) M.empty}
\end{Highlighting}
\end{Shaded}

Only slightly more involved, but still pretty readable, right? We find
out if things have flooded, and our target channels will be just the
original source if true (a message as a reprimand); otherwise, all the
channels in \texttt{chans}. If they have flooded, then our \texttt{outB}
(blip stream of \texttt{{[}Message{]}} to send to each room) will just
be \texttt{{[}"No\ flooding!"{]}} if yes, or the actual announcement
otherwise.

Finally, our \texttt{Blip\ OutMessages} will be the \texttt{OutMessage}
formed by associating all of the channels in \texttt{targetChans} with
the message in \texttt{outB}\ldots{}emitting whenever \texttt{outB}
emits.

Note here that we use \texttt{(\textless{}\$)} from the \texttt{Functor}
instance of blip streams. \texttt{x\ \textless{}\$\ fooB} is a new blip
stream that emits whenever \texttt{fooB} emits\ldots{}but instead
\emph{replaces the emitted value}. So for
\texttt{4\ \textless{}\$\ fooB}, if \texttt{fooB} emits with
\texttt{"hello"}, \texttt{4\ \textless{}\$\ fooB} emits with \texttt{4}.
Emit at the same time, but pop out the value and put in your own.

Finally we use \texttt{fromBlips}, which we met before in the definition
of \texttt{perRoom}: the output is the \texttt{OutMessage} in
\texttt{outMsgsB} whenever \texttt{outMsgsB} \emph{does} emit\ldots{}or
it's \texttt{mempty} (the empty map) when it doesn't.

\section{Wrapping it all up}\label{wrapping-it-all-up}

We have three nice modules now. Now let's wrap it all together.

First, if you've been following around, you might have noticed that we
needed \texttt{Serialize} instances (from the \emph{cereal} library) for
\texttt{UTCTime} and \texttt{Day} in order for \texttt{trackSeens} and
\texttt{newDayBlips} to serialize properly. We can just write really
rough versions of them for now for demonstration purposes:

\begin{Shaded}
\begin{Highlighting}[]
\CommentTok{-- source: https://github.com/mstksg/blog/tree/develop/code-samples/auto/chatbot.hs#L201-207}
\KeywordTok{instance} \DataTypeTok{Serialize} \DataTypeTok{UTCTime} \KeywordTok{where}
    \NormalTok{get }\FunctionTok{=} \NormalTok{read }\FunctionTok{<$>} \NormalTok{get      }\CommentTok{-- haha don't do this in real life.}
    \NormalTok{put }\FunctionTok{=} \NormalTok{put }\FunctionTok{.} \NormalTok{show}

\KeywordTok{instance} \DataTypeTok{Serialize} \DataTypeTok{Day} \KeywordTok{where}
    \NormalTok{get }\FunctionTok{=} \DataTypeTok{ModifiedJulianDay} \FunctionTok{<$>} \NormalTok{get}
    \NormalTok{put }\FunctionTok{=} \NormalTok{put }\FunctionTok{.} \NormalTok{toModifiedJulianDay}
\end{Highlighting}
\end{Shaded}

And, writing \texttt{chatBot}:

\begin{Shaded}
\begin{Highlighting}[]
\CommentTok{-- source: https://github.com/mstksg/blog/tree/develop/code-samples/auto/chatbot.hs#L54-59}
\OtherTok{chatBot ::} \DataTypeTok{MonadIO} \NormalTok{m }\OtherTok{=>} \DataTypeTok{ChatBot} \NormalTok{m}
\NormalTok{chatBot }\FunctionTok{=} \NormalTok{serializing' }\StringTok{"chatbot.dat"}
        \FunctionTok{.} \NormalTok{mconcat }\FunctionTok{$} \NormalTok{[ perRoom seenBot}
                    \NormalTok{, perRoom repBot}
                    \NormalTok{, announceBot [}\StringTok{"#jlebot-test"}\NormalTok{]}
                    \NormalTok{]}
\end{Highlighting}
\end{Shaded}

Or, to future-proof, in case we foresee adding new modules:

\begin{Shaded}
\begin{Highlighting}[]
\CommentTok{-- source: https://github.com/mstksg/blog/tree/develop/code-samples/auto/chatbot.hs#L61-65}
\OtherTok{chatBot' ::} \DataTypeTok{MonadIO} \NormalTok{m }\OtherTok{=>} \DataTypeTok{ChatBot} \NormalTok{m}
\NormalTok{chatBot' }\FunctionTok{=} \NormalTok{mconcat [ perRoom }\FunctionTok{.} \NormalTok{serializing' }\StringTok{"seens.dat"} \FunctionTok{$} \NormalTok{seenBot}
                   \NormalTok{, perRoom }\FunctionTok{.} \NormalTok{serializing' }\StringTok{"reps.dat"}  \FunctionTok{$} \NormalTok{repBot}
                   \NormalTok{,           serializing' }\StringTok{"anns.dat"}  \FunctionTok{$} \NormalTok{announceBot [}\StringTok{"#jlebot-test"}\NormalTok{]}
                   \NormalTok{]}
\end{Highlighting}
\end{Shaded}

And\ldots{}that's it!

\section{Fin}\label{fin}

Hopefully, going over this project, you're starting to see some common
and powerful idioms and tools. I hope that a clear picture of how to
approach and finish a program with the \emph{auto} library
looks\ldots{}and how beneficial the platform and what it offers is to
streamlining the development process.

Also, hopefully the ``declarative'' nature of everything is apparent.
Especially for \emph{proc} blocks\ldots{}everything just ``looks like''
a graph of relationships. This quantity is related to this quantitiy in
this way, this quantity is related to that in that way, etc. It looks
like you're just specifying a graph of relationships, which is really
what the core of \emph{auto} is all about. We assemble complex
relationships by putting together small, simple relationships.

Note that we didn't just implement ``easy''
modules/components\ldots{}these are actual working components that you
might see in real bots, and not just toy ones.

Where can we go from here? Well, you might actually want to maybe write
``subscription'' \texttt{Auto}s that are updated every minute or so:

\begin{Shaded}
\begin{Highlighting}[]
\KeywordTok{type} \DataTypeTok{ChronBot} \NormalTok{m }\FunctionTok{=} \DataTypeTok{Auto} \NormalTok{m }\DataTypeTok{UTCTime} \DataTypeTok{OutMessages}
\end{Highlighting}
\end{Shaded}

You call them every minute with the type, and it's allowed to react with
the the time and output an \texttt{OutMessages}. You can use this bot to
implement things like rss feed watchers/subscribers, for instance.

Then, instead of using an input channel waiting for \texttt{InMessage},
you might wait for \texttt{Either\ InMessage\ UTCTime}\ldots{}drop in
\texttt{Left\ im} whenever you get a message, and \texttt{Right\ time}
from a thread that just waits a minute and repeatedly throws in times.

We can do this with minimal extra work by using the
\texttt{(\textbar{}\textbar{}\textbar{})} combinator from
\texttt{Control.Auto}:

\begin{Shaded}
\begin{Highlighting}[]
\OtherTok{(|||) ::} \DataTypeTok{Auto} \NormalTok{m a c }\OtherTok{->} \DataTypeTok{Auto} \NormalTok{m b c }\OtherTok{->} \DataTypeTok{Auto} \NormalTok{m (}\DataTypeTok{Either} \NormalTok{a         b      ) c}
\OtherTok{(|||) ::} \DataTypeTok{ChatBot} \NormalTok{m  }\OtherTok{->} \DataTypeTok{ChronBot} \NormalTok{m }\OtherTok{->} \DataTypeTok{Auto} \NormalTok{m (}\DataTypeTok{Either} \DataTypeTok{InMessage} \DataTypeTok{UTCTime}\NormalTok{) }\DataTypeTok{OutMessages}
\end{Highlighting}
\end{Shaded}

And\ldots{}you get it all for free! No extra work. Now both the
\texttt{ChatBot} and the \texttt{ChronBot} will wait on the input
stream, and the \texttt{Left}s will be fed to the \texttt{ChatBot} and
the \texttt{Right}s will be fed to the \texttt{ChronBot}.

Anyway, this post is long enough. Have fun exploring \emph{auto} on your
own; I'm always happy to hear about any project you might be working on!
You can find me on twitter as \href{https://twitter.com/mstk}{mstk}. If
you have any questions or comments/suggestions, feel free to leave a
comment down below or drop by freenode's \emph{\#haskell-auto} or
\emph{\#haskell-game}, where I go by \emph{jle`}! And, as always, happy
Haskelling!

\end{document}
