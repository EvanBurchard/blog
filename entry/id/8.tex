\documentclass[]{article}
\usepackage{lmodern}
\usepackage{amssymb,amsmath}
\usepackage{ifxetex,ifluatex}
\usepackage{fixltx2e} % provides \textsubscript
\ifnum 0\ifxetex 1\fi\ifluatex 1\fi=0 % if pdftex
  \usepackage[T1]{fontenc}
  \usepackage[utf8]{inputenc}
\else % if luatex or xelatex
  \ifxetex
    \usepackage{mathspec}
    \usepackage{xltxtra,xunicode}
  \else
    \usepackage{fontspec}
  \fi
  \defaultfontfeatures{Mapping=tex-text,Scale=MatchLowercase}
  \newcommand{\euro}{€}
\fi
% use upquote if available, for straight quotes in verbatim environments
\IfFileExists{upquote.sty}{\usepackage{upquote}}{}
% use microtype if available
\IfFileExists{microtype.sty}{\usepackage{microtype}}{}
\usepackage[margin=1in]{geometry}
\usepackage{color}
\usepackage{fancyvrb}
\newcommand{\VerbBar}{|}
\newcommand{\VERB}{\Verb[commandchars=\\\{\}]}
\DefineVerbatimEnvironment{Highlighting}{Verbatim}{commandchars=\\\{\}}
% Add ',fontsize=\small' for more characters per line
\newenvironment{Shaded}{}{}
\newcommand{\KeywordTok}[1]{\textcolor[rgb]{0.00,0.44,0.13}{\textbf{{#1}}}}
\newcommand{\DataTypeTok}[1]{\textcolor[rgb]{0.56,0.13,0.00}{{#1}}}
\newcommand{\DecValTok}[1]{\textcolor[rgb]{0.25,0.63,0.44}{{#1}}}
\newcommand{\BaseNTok}[1]{\textcolor[rgb]{0.25,0.63,0.44}{{#1}}}
\newcommand{\FloatTok}[1]{\textcolor[rgb]{0.25,0.63,0.44}{{#1}}}
\newcommand{\ConstantTok}[1]{\textcolor[rgb]{0.53,0.00,0.00}{{#1}}}
\newcommand{\CharTok}[1]{\textcolor[rgb]{0.25,0.44,0.63}{{#1}}}
\newcommand{\SpecialCharTok}[1]{\textcolor[rgb]{0.25,0.44,0.63}{{#1}}}
\newcommand{\StringTok}[1]{\textcolor[rgb]{0.25,0.44,0.63}{{#1}}}
\newcommand{\VerbatimStringTok}[1]{\textcolor[rgb]{0.25,0.44,0.63}{{#1}}}
\newcommand{\SpecialStringTok}[1]{\textcolor[rgb]{0.73,0.40,0.53}{{#1}}}
\newcommand{\ImportTok}[1]{{#1}}
\newcommand{\CommentTok}[1]{\textcolor[rgb]{0.38,0.63,0.69}{\textit{{#1}}}}
\newcommand{\DocumentationTok}[1]{\textcolor[rgb]{0.73,0.13,0.13}{\textit{{#1}}}}
\newcommand{\AnnotationTok}[1]{\textcolor[rgb]{0.38,0.63,0.69}{\textbf{\textit{{#1}}}}}
\newcommand{\CommentVarTok}[1]{\textcolor[rgb]{0.38,0.63,0.69}{\textbf{\textit{{#1}}}}}
\newcommand{\OtherTok}[1]{\textcolor[rgb]{0.00,0.44,0.13}{{#1}}}
\newcommand{\FunctionTok}[1]{\textcolor[rgb]{0.02,0.16,0.49}{{#1}}}
\newcommand{\VariableTok}[1]{\textcolor[rgb]{0.10,0.09,0.49}{{#1}}}
\newcommand{\ControlFlowTok}[1]{\textcolor[rgb]{0.00,0.44,0.13}{\textbf{{#1}}}}
\newcommand{\OperatorTok}[1]{\textcolor[rgb]{0.40,0.40,0.40}{{#1}}}
\newcommand{\BuiltInTok}[1]{{#1}}
\newcommand{\ExtensionTok}[1]{{#1}}
\newcommand{\PreprocessorTok}[1]{\textcolor[rgb]{0.74,0.48,0.00}{{#1}}}
\newcommand{\AttributeTok}[1]{\textcolor[rgb]{0.49,0.56,0.16}{{#1}}}
\newcommand{\RegionMarkerTok}[1]{{#1}}
\newcommand{\InformationTok}[1]{\textcolor[rgb]{0.38,0.63,0.69}{\textbf{\textit{{#1}}}}}
\newcommand{\WarningTok}[1]{\textcolor[rgb]{0.38,0.63,0.69}{\textbf{\textit{{#1}}}}}
\newcommand{\AlertTok}[1]{\textcolor[rgb]{1.00,0.00,0.00}{\textbf{{#1}}}}
\newcommand{\ErrorTok}[1]{\textcolor[rgb]{1.00,0.00,0.00}{\textbf{{#1}}}}
\newcommand{\NormalTok}[1]{{#1}}
\ifxetex
  \usepackage[setpagesize=false, % page size defined by xetex
              unicode=false, % unicode breaks when used with xetex
              xetex]{hyperref}
\else
  \usepackage[unicode=true]{hyperref}
\fi
\hypersetup{breaklinks=true,
            bookmarks=true,
            pdfauthor={Justin Le},
            pdftitle={log.sh: Lightweight Command Line Note \& Logging},
            colorlinks=true,
            citecolor=blue,
            urlcolor=blue,
            linkcolor=magenta,
            pdfborder={0 0 0}}
\urlstyle{same}  % don't use monospace font for urls
% Make links footnotes instead of hotlinks:
\renewcommand{\href}[2]{#2\footnote{\url{#1}}}
\setlength{\parindent}{0pt}
\setlength{\parskip}{6pt plus 2pt minus 1pt}
\setlength{\emergencystretch}{3em}  % prevent overfull lines
\setcounter{secnumdepth}{0}

\title{log.sh: Lightweight Command Line Note \& Logging}
\author{Justin Le}
\date{October 15, 2013}

\begin{document}
\maketitle

\emph{Originally posted on
\textbf{\href{http://home.jle0.com:4111/entry/log-sh-lightweight-command-line-note-logging.html}{in
Code}}.}

What do you use to send off quick one-off notes and logs about a project you are
working on? Found a nice link to a resource you'll want to look up
later\ldots{}want to jot down a sudden realization?

Maybe you use some external note-taking software, like \emph{Evernote}. But
wouldn't it be nice to have something that is completely in the command line? Do
you really need to fire up an entire GUI just to write down one line, put down
one link? And do you really need these notes to all be thrown in with your
others?

You might be using a command line interface to a larger note-taking system like
\emph{\href{http://geeknote.me/}{geeknote}}. But it's kind of a hassle to open
up an entire text editor every time you want to make a small one-liner. Doesn't
quite meld with the \href{http://www.faqs.org/docs/artu/ch01s06.html}{Unix
philosophy}. Maybe you are comfortable simply appending to a text file with
\texttt{\textgreater{}\textgreater{}}\ldots{}but what if you want to add things
like timestamps?

Here's introducing
\textbf{\emph{\href{https://github.com/mstksg/log.sh}{log.sh}}}.

\section{\texorpdfstring{\href{https://github.com/mstksg/log.sh}{Log.sh}}{Log.sh}}\label{log.shlog.sh}

\emph{\href{https://github.com/mstksg/log.sh}{log.sh}} is intended for these use
cases:

\begin{itemize}
\tightlist
\item
  Project-based notes

  \begin{itemize}
  \tightlist
  \item
    Quick links to resources, references
  \item
    Small local project TODO's
  \item
    Reminders and gotchas
  \item
    Logging progress, short micro-journaling to record check points in progress.
  \end{itemize}
\item
  Simple quick references (in the home directory)

  \begin{itemize}
  \tightlist
  \item
    Reminders on system configuration todo's
  \item
    Refreshers on simple tasks that you only do a few times a year
  \end{itemize}
\end{itemize}

Most of all, it is meant to be quick, fast, efficient, streamlined, simple.

\subsection{Usage}\label{usage}

\subsubsection{Entering a note}\label{entering-a-note}

Don't believe me? Here's how you enter in a note:

\begin{Shaded}
\begin{Highlighting}[]
\NormalTok{$ }\ExtensionTok{log.sh} \NormalTok{hey, this is a note!}
\end{Highlighting}
\end{Shaded}

(Note no need to surround your log with quotes or anything.)

That command will write the line

\begin{verbatim}
[Sun Sep 29 16:07:21 PDT 2013]  hey, this is a note!
\end{verbatim}

to the active log file. As a cute trick, you can specify the flag \texttt{-t}

\begin{Shaded}
\begin{Highlighting}[]
\NormalTok{$ }\ExtensionTok{log.sh} \NormalTok{-t buy milk}
\end{Highlighting}
\end{Shaded}

which will handily add a cute \texttt{{[}\ {]}} before the item like so:

\begin{verbatim}
[Sun Sep 29 16:14:03 PDT 2013]  [ ] buy milk
\end{verbatim}

\subsubsection{Editing your log file}\label{editing-your-log-file}

The little \texttt{{[}\ {]}} is handy because you can open up the log file in a
text editor of your choice and ``check'' it off.

That's actually very easy, because

\begin{Shaded}
\begin{Highlighting}[]
\NormalTok{$ }\ExtensionTok{log.sh} \NormalTok{-e}
\end{Highlighting}
\end{Shaded}

will open the active log file in your favorite editor, specified by the
\texttt{\$EDITOR} environment variable.

\subsubsection{The Log File}\label{the-log-file}

What is the ``active'' log file? Where is this log being stored? Ideally, it is
either in the root directory of the relevant project, or in your home folder.

To create a log file in your current working directory, enter

\begin{Shaded}
\begin{Highlighting}[]
\NormalTok{$ }\ExtensionTok{log.sh} \NormalTok{-c}
\end{Highlighting}
\end{Shaded}

This'll create a file \texttt{.log.log} in the directory.

Now, if you are in any subdirectory, the default behavior is that
\emph{\href{https://github.com/mstksg/log.sh}{log.sh}} will first check the
current directory for a log file; if it doesn't find it, it goes up a directory
and checks there, checking up and up until it finds a valid log file. That file
is the ``active'' one, where all of your adds and edits will refer to.

\subsubsection{Log Contexts}\label{log-contexts}

If you try to enter a note from a subdirectory, you'll get something cool:

\begin{verbatim}
[Sun Sep 29 16:10:38 PDT 2013]  (./subdir) logging from ./subdir!
\end{verbatim}

\emph{\href{https://github.com/mstksg/log.sh}{log.sh}} will automatically
include the context of your log in your note body.

\subsection{Installation}\label{installation}

Didn't want to bore you right off the bat with implementation/installation
details. But here it is. The entire thing is open source, and written in bash.

Install by either cloning the github repo or downloading
\href{https://github.com/mstksg/log.sh/releases}{the latest release}. Put the
file \texttt{log.sh} into a directory in your \texttt{\$PATH}. If you wish,
alias it to something short, like \texttt{l} or \texttt{n}.

That should be it!

\subsection{That's it!}\label{thats-it}

That's really all there is to it! You can customize the filename of the log file
created/searched for, or you can even specify the exact path of the log file you
want to append to or edit using command line flags. More detail on how to do
this in the documentation:

\begin{Shaded}
\begin{Highlighting}[]
\NormalTok{$ }\ExtensionTok{log.sh} \NormalTok{-h}
\end{Highlighting}
\end{Shaded}

Basically, the entire thing is meant to be as frictionless, fast, and
thoughtless as possible. Enter in small notes to reference later in only the
time it takes for you to actually write the note --- no need to mungle around
with text editors and picking which file to append to and dealing with
timestamps. \emph{\href{https://github.com/mstksg/log.sh}{log.sh}} has got you
covered!

I do recommend, if you use this, aliasing the command to something short. I
personally use \texttt{n}, so I write notes by saying:

\begin{Shaded}
\begin{Highlighting}[]
\NormalTok{$ }\ExtensionTok{n} \NormalTok{goodbye!}
\end{Highlighting}
\end{Shaded}

Hopefully this script ends up being as useful to you as it has to me. Feel free
to leave any comments on questions/bugs/improvements, and I'm always happy to
take contributions and pull requests.

\end{document}
