\documentclass[]{article}
\usepackage{lmodern}
\usepackage{amssymb,amsmath}
\usepackage{ifxetex,ifluatex}
\usepackage{fixltx2e} % provides \textsubscript
\ifnum 0\ifxetex 1\fi\ifluatex 1\fi=0 % if pdftex
  \usepackage[T1]{fontenc}
  \usepackage[utf8]{inputenc}
\else % if luatex or xelatex
  \ifxetex
    \usepackage{mathspec}
    \usepackage{xltxtra,xunicode}
  \else
    \usepackage{fontspec}
  \fi
  \defaultfontfeatures{Mapping=tex-text,Scale=MatchLowercase}
  \newcommand{\euro}{€}
\fi
% use upquote if available, for straight quotes in verbatim environments
\IfFileExists{upquote.sty}{\usepackage{upquote}}{}
% use microtype if available
\IfFileExists{microtype.sty}{\usepackage{microtype}}{}
\usepackage[margin=1in]{geometry}
\usepackage{color}
\usepackage{fancyvrb}
\newcommand{\VerbBar}{|}
\newcommand{\VERB}{\Verb[commandchars=\\\{\}]}
\DefineVerbatimEnvironment{Highlighting}{Verbatim}{commandchars=\\\{\}}
% Add ',fontsize=\small' for more characters per line
\newenvironment{Shaded}{}{}
\newcommand{\KeywordTok}[1]{\textcolor[rgb]{0.00,0.44,0.13}{\textbf{{#1}}}}
\newcommand{\DataTypeTok}[1]{\textcolor[rgb]{0.56,0.13,0.00}{{#1}}}
\newcommand{\DecValTok}[1]{\textcolor[rgb]{0.25,0.63,0.44}{{#1}}}
\newcommand{\BaseNTok}[1]{\textcolor[rgb]{0.25,0.63,0.44}{{#1}}}
\newcommand{\FloatTok}[1]{\textcolor[rgb]{0.25,0.63,0.44}{{#1}}}
\newcommand{\ConstantTok}[1]{\textcolor[rgb]{0.53,0.00,0.00}{{#1}}}
\newcommand{\CharTok}[1]{\textcolor[rgb]{0.25,0.44,0.63}{{#1}}}
\newcommand{\SpecialCharTok}[1]{\textcolor[rgb]{0.25,0.44,0.63}{{#1}}}
\newcommand{\StringTok}[1]{\textcolor[rgb]{0.25,0.44,0.63}{{#1}}}
\newcommand{\VerbatimStringTok}[1]{\textcolor[rgb]{0.25,0.44,0.63}{{#1}}}
\newcommand{\SpecialStringTok}[1]{\textcolor[rgb]{0.73,0.40,0.53}{{#1}}}
\newcommand{\ImportTok}[1]{{#1}}
\newcommand{\CommentTok}[1]{\textcolor[rgb]{0.38,0.63,0.69}{\textit{{#1}}}}
\newcommand{\DocumentationTok}[1]{\textcolor[rgb]{0.73,0.13,0.13}{\textit{{#1}}}}
\newcommand{\AnnotationTok}[1]{\textcolor[rgb]{0.38,0.63,0.69}{\textbf{\textit{{#1}}}}}
\newcommand{\CommentVarTok}[1]{\textcolor[rgb]{0.38,0.63,0.69}{\textbf{\textit{{#1}}}}}
\newcommand{\OtherTok}[1]{\textcolor[rgb]{0.00,0.44,0.13}{{#1}}}
\newcommand{\FunctionTok}[1]{\textcolor[rgb]{0.02,0.16,0.49}{{#1}}}
\newcommand{\VariableTok}[1]{\textcolor[rgb]{0.10,0.09,0.49}{{#1}}}
\newcommand{\ControlFlowTok}[1]{\textcolor[rgb]{0.00,0.44,0.13}{\textbf{{#1}}}}
\newcommand{\OperatorTok}[1]{\textcolor[rgb]{0.40,0.40,0.40}{{#1}}}
\newcommand{\BuiltInTok}[1]{{#1}}
\newcommand{\ExtensionTok}[1]{{#1}}
\newcommand{\PreprocessorTok}[1]{\textcolor[rgb]{0.74,0.48,0.00}{{#1}}}
\newcommand{\AttributeTok}[1]{\textcolor[rgb]{0.49,0.56,0.16}{{#1}}}
\newcommand{\RegionMarkerTok}[1]{{#1}}
\newcommand{\InformationTok}[1]{\textcolor[rgb]{0.38,0.63,0.69}{\textbf{\textit{{#1}}}}}
\newcommand{\WarningTok}[1]{\textcolor[rgb]{0.38,0.63,0.69}{\textbf{\textit{{#1}}}}}
\newcommand{\AlertTok}[1]{\textcolor[rgb]{1.00,0.00,0.00}{\textbf{{#1}}}}
\newcommand{\ErrorTok}[1]{\textcolor[rgb]{1.00,0.00,0.00}{\textbf{{#1}}}}
\newcommand{\NormalTok}[1]{{#1}}
\ifxetex
  \usepackage[setpagesize=false, % page size defined by xetex
              unicode=false, % unicode breaks when used with xetex
              xetex]{hyperref}
\else
  \usepackage[unicode=true]{hyperref}
\fi
\hypersetup{breaklinks=true,
            bookmarks=true,
            pdfauthor={Justin Le},
            pdftitle={Shake: Task Automation and Scripting in Haskell},
            colorlinks=true,
            citecolor=blue,
            urlcolor=blue,
            linkcolor=magenta,
            pdfborder={0 0 0}}
\urlstyle{same}  % don't use monospace font for urls
% Make links footnotes instead of hotlinks:
\renewcommand{\href}[2]{#2\footnote{\url{#1}}}
\setlength{\parindent}{0pt}
\setlength{\parskip}{6pt plus 2pt minus 1pt}
\setlength{\emergencystretch}{3em}  % prevent overfull lines
\setcounter{secnumdepth}{0}

\title{Shake: Task Automation and Scripting in Haskell}
\author{Justin Le}
\date{September 17, 2013}

\begin{document}
\maketitle

\emph{Originally posted on
\textbf{\href{http://home.jle0.com:4111/entry/shake-task-automation-and-scripting-in-haskell.html}{in
Code}}.}

As someone who comes from a background in ruby and \emph{rake}, I'm used to
powerful task management systems with expressive dependency. \emph{Make} is a
favorite tool of mine when I'm working on projects with people who don't use
ruby, and when I'm working on ruby projects I never go far without starting a
good Rakefile. The two tools provided a perfect DSL for setting up systems of
tasks that had complicated file and task dependencies.

As I was starting to learn Haskell and building larger-scale Haskell projects, I
began to look for alternatives in Haskell. Was there a Haskell counterpart to
Ruby's \href{http://rake.rubyforge.org/}{\emph{rake}}, Node's
\href{https://github.com/mde/jake}{\emph{jake}}? (Not to mention the tools of
slightly different philosophy \href{http://gruntjs.com/}{\emph{grunt}} and
\href{http://ant.apache.org/}{\emph{ant}})

It turns out that by far the most established answer is a library known as
\href{http://hackage.haskell.org/package/shake}{\emph{Shake}} (maintained by the
prolific Neil Mitchell of \href{http://haskell.org/hoogle}{\emph{hoogle}} fame
and much more). So far it's served me pretty well. Its documentation is written
from the perspective of chiefly using it as a build tool (more ``make'' than
``rake''), so if you're looking to use it as a task management system, you might
have to do some digging. Hopefully this post can help you get started.

I also go over the core concepts of a task management system, so I assume no
knowledge of \emph{make}; this post therefore should also be a good introduction
to starting with any sort of task management system.

\section{Our Sample Project}\label{our-sample-project}

Our sample project is going to be a report build system that builds reports
written in markdown with \href{http://johnmacfarlane.net/pandoc/}{pandoc} into
html, pdf, and doc formats. This is honestly one of my most common use cases for
\emph{make}, so porting it all to \emph{shake} will be something useful for me.

The final directory structure will look like this:

\begin{quote}
\begin{itemize}
\tightlist
\item
  img

  \begin{itemize}
  \tightlist
  \item
    img1.jpg
  \item
    img2.jpg
  \end{itemize}
\item
  out

  \begin{itemize}
  \tightlist
  \item
    report.doc
  \item
    report.html
  \item
    report.pdf
  \end{itemize}
\item
  src

  \begin{itemize}
  \tightlist
  \item
    report.md
  \end{itemize}
\item
  css

  \begin{itemize}
  \tightlist
  \item
    report.css
  \end{itemize}
\item
  Shakefile
\end{itemize}
\end{quote}

When we run \texttt{shake}, we want to build \texttt{report.doc} and
\texttt{report.pdf} if \texttt{report.md} or any of the images have changed, and
\texttt{report.html} if \texttt{report.md}, \texttt{report.css}, or any of the
images have changed.

Furthermore, \texttt{img2.jpg} actually comes from online, and requires us to
re-download it every time we compile to make sure it is up to date.

\section{Setup}\label{setup}

\subsection{Installing Shake}\label{installing-shake}

Installing \emph{shake} is as simple as installing any other cabal package:

\begin{Shaded}
\begin{Highlighting}[]
\NormalTok{$ }\KeywordTok{cabal} \NormalTok{update}
\NormalTok{$ }\KeywordTok{cabal} \NormalTok{install shake}
\end{Highlighting}
\end{Shaded}

I'll will be using \texttt{shake-0.10.6} for this post.

\subsection{Setting up the Shakefile}\label{setting-up-the-shakefile}

We set up our Shakefile with a simple scaffold:

\begin{Shaded}
\begin{Highlighting}[]
\CommentTok{-- Shakefile}

\KeywordTok{import }\DataTypeTok{Development.Shake}

\NormalTok{opts }\FunctionTok{=} \NormalTok{shakeOptions \{ shakeFiles    }\FunctionTok{=} \StringTok{".shake/"} \NormalTok{\}        }\CommentTok{-- 1}

\NormalTok{(}\FunctionTok{~>}\NormalTok{) }\FunctionTok{=} \NormalTok{phony                                             }\CommentTok{-- 2}
                                                         \CommentTok{-- (obsolete)}

\OtherTok{main ::} \DataTypeTok{IO} \NormalTok{()}
\NormalTok{main }\FunctionTok{=} \NormalTok{shakeArgs opts }\FunctionTok{$} \KeywordTok{do}
    \NormalTok{want []}

    \StringTok{"clean"} \FunctionTok{~>} \NormalTok{removeFilesAfter }\StringTok{".shake"} \NormalTok{[}\StringTok{"//*"}\NormalTok{]}
\end{Highlighting}
\end{Shaded}

On my machine I've set this up to be generated by a
\href{https://gist.github.com/mstksg/6588764}{bash script} called ``shakeup'',
so I can start a project up on a Shakefile by simply typing \texttt{shakeup} at
the project root.

Some notes:

\begin{enumerate}
\def\labelenumi{\arabic{enumi}.}
\item
  Store shake's metadata files to the folder \texttt{.shake/}. This differs from
  the default behavior, where all files would be saved to the root directory
  with \texttt{.shake} as a filename prefix.
\item
  I've aliased the operator \texttt{\textasciitilde{}\textgreater{}} for
  \texttt{phony} to allow for a more expressive infix notation --- more on this
  later. I've submitted a patch to the project and it should be included in the
  next cabal release.

  \textbf{Edit}: As of the 0.10.7 release of \emph{Shake}, this is no longer
  needed, as \texttt{\textasciitilde{}\textgreater{}} is included in the
  library.
\end{enumerate}

\section{What is a Rule?}\label{what-is-a-rule}

If you haven't used \emph{make} before, it is important that you understand the
key concepts before moving on.

A task management system/build system is a system that works to ensure that all
files in the project are ``up to date''. In our case, our system will ensure
that the files in the \texttt{out} directory are up to date.

In order to do this, files are given ``rules''. Rules specify:

\begin{enumerate}
\def\labelenumi{\arabic{enumi}.}
\item
  What other files/rules this file ``depends'' on
\item
  Instructions to execute to make this file up to date (or to create the file),
  if it is not already up to date or created.
\end{enumerate}

A file or rule is out of date if any of its dependencies are out of date
\textbf{or} if the file it indicates is either not created or has been updated
since the last time the task management system has run. When this happens, the
guilty dependencies are updated using their own rules. Afterwards, the file's
own instructions are executed.

If a file has no rule, ``out of date'' simply means that it has been updated or
changed since the last time the task management system has run, or it does not
exist. If it has, then all files or rules that depend on it are also out of
date.

A good task management system is smart enough to keep track of what is up to
date and what isn't. If multiple rules all have one dependency, that dependency
might be checked and updated every single time. For example, all of our builds
in this sample project require \texttt{img2.jpg} to be downloaded afresh from
online. A naive build system might re-download \texttt{img2.jpg} for every
single build, instead of once for all three.

\section{File Rules}\label{file-rules}

Let's set up \texttt{src/report.md} with a simple markdown document on our new
project:

\begin{Shaded}
\begin{Highlighting}[]
\NormalTok{<!--}\CommentTok{ src/report.md -->}

\NormalTok{Report}
\NormalTok{======}

\NormalTok{This is a report.  Render me!}

\AlertTok{![first image](img/img1.jpg)}
\AlertTok{![second image](img/img2.jpg)}
\end{Highlighting}
\end{Shaded}

Our project tree should look like this at this point:

\begin{quote}
\begin{itemize}
\tightlist
\item
  img

  \begin{itemize}
  \tightlist
  \item
    img1.jpg
  \item
    img2.jpg
  \end{itemize}
\item
  out
\item
  src

  \begin{itemize}
  \tightlist
  \item
    report.md
  \end{itemize}
\item
  template
\item
  Shakefile
\end{itemize}
\end{quote}

Let's set up our first rule -- rendering \texttt{out/report.doc} if
\texttt{report.md} has changed.

\begin{Shaded}
\begin{Highlighting}[]
\StringTok{"out/report.doc"} \FunctionTok{*>} \NormalTok{\textbackslash{}f }\OtherTok{->} \KeywordTok{do}
    \NormalTok{need [}\StringTok{"src/report.md"}\NormalTok{,}\StringTok{"img/img1.jpg"}\NormalTok{,}\StringTok{"img/img2.jpg"}\NormalTok{]}
    \NormalTok{cmd }\StringTok{"pandoc"} \NormalTok{[ }\StringTok{"src/report.md"}\NormalTok{, }\StringTok{"-o"}\NormalTok{, f ]}
\end{Highlighting}
\end{Shaded}

This is equivalent to the Makefile rule:

\begin{Shaded}
\begin{Highlighting}[]
\DecValTok{out/report.doc:}\DataTypeTok{ src/report.md}
	\NormalTok{pandoc src/report.md -o out/report.doc}
\end{Highlighting}
\end{Shaded}

The operator \texttt{*\textgreater{}} attaches an
\href{http://hackage.haskell.org/packages/archive/shake/0.10.6/doc/html/Development-Shake.html\#t:Action}{\texttt{Action}}
(with a parameter) to a
\href{http://hackage.haskell.org/packages/archive/shake/0.10.6/doc/html/Development-Shake.html\#t:FilePattern}{\texttt{FilePattern}}
(a string) -- that is, when \emph{shake} decides that it needs that specified
file on the left hand side to be up to date, it runs the action on the right
hand side with that filename as a parameter.

To be clear, the right hand side is of type:

\begin{Shaded}
\begin{Highlighting}[]
\OtherTok{rightHandSide ::} \DataTypeTok{FilePattern} \OtherTok{->} \DataTypeTok{Action} \NormalTok{()}
\end{Highlighting}
\end{Shaded}

where the \texttt{FilePattern} is the filename of the file that is being
``needed''.

The \texttt{need} function specifies all of the dependencies of that action. If
\emph{shake} decides it needs \texttt{out/report.doc} to be up to date,
\texttt{need} tells it that it first needs \texttt{src/report.md} and the images
to be up to date -- or rather, that \texttt{out/report.doc} is only out of date
if \texttt{src/report.md} or the images are out of date, or have changed since
the last build.

With this in mind, let us write the rest of our file rules:

\begin{Shaded}
\begin{Highlighting}[]
\CommentTok{-- Shakefile}

\StringTok{"out/report.doc"} \FunctionTok{*>} \NormalTok{\textbackslash{}f }\OtherTok{->} \KeywordTok{do}
    \NormalTok{need [}\StringTok{"src/report.md"}\NormalTok{,}\StringTok{"img/img1.jpg"}\NormalTok{,}\StringTok{"img/img2.jpg"}\NormalTok{]}
    \NormalTok{cmd }\StringTok{"pandoc"} \NormalTok{[ }\StringTok{"src/report.md"}\NormalTok{, }\StringTok{"-o"}\NormalTok{, f ]}

\StringTok{"out/report.pdf"} \FunctionTok{*>} \NormalTok{\textbackslash{}f }\OtherTok{->} \KeywordTok{do}
    \NormalTok{need [}\StringTok{"src/report.md"}\NormalTok{,}\StringTok{"img/img1.jpg"}\NormalTok{,}\StringTok{"img/img2.jpg"}\NormalTok{]}
    \NormalTok{cmd }\StringTok{"pandoc"} \NormalTok{[ }\StringTok{"src/report.md"}\NormalTok{, }\StringTok{"-o"}\NormalTok{, f, }\StringTok{"-V"}\NormalTok{, }\StringTok{"links-as-notes"} \NormalTok{]}

\StringTok{"out/report.html"} \FunctionTok{*>} \NormalTok{\textbackslash{}f }\OtherTok{->} \KeywordTok{do}
    \NormalTok{need [ }\StringTok{"src/report.md"}
         \NormalTok{, }\StringTok{"img/img1.jpg"}
         \NormalTok{, }\StringTok{"img/img2.jpg"}
         \NormalTok{, }\StringTok{"css/report.css"} \NormalTok{]}
    \NormalTok{cmd }\StringTok{"pandoc"} \NormalTok{[ }\StringTok{"src/report.md"}\NormalTok{, }\StringTok{"-o"}\NormalTok{, f, }\StringTok{"-c"}\NormalTok{, }\StringTok{"css/report.css"}\NormalTok{, }\StringTok{"-S"} \NormalTok{]}

\StringTok{"img/img2.jpg"} \FunctionTok{*>} \NormalTok{\textbackslash{}f }\OtherTok{->} \KeywordTok{do}
    \NormalTok{cmd }\StringTok{"wget"} \NormalTok{[ }\StringTok{"http://example.com/img2.jpg"}\NormalTok{, }\StringTok{"-O"}\NormalTok{, f ]}
\end{Highlighting}
\end{Shaded}

And that is it!

\section{Running Shake}\label{running-shake}

How do we tell \emph{shake} what file it is that we want to be up to date? We
specify this by modifying the line \texttt{want\ {[}{]}}:

\begin{Shaded}
\begin{Highlighting}[]
\NormalTok{want [}\StringTok{"out/report.doc"}\NormalTok{,}\StringTok{"out/report.pdf"}\NormalTok{,}\StringTok{"out/report.html"}\NormalTok{]}
\end{Highlighting}
\end{Shaded}

That tells \emph{shake} that when we run \texttt{main} with no arguments, we
want those three files to be checked to be up to date.

Now, to wrap it all together, we run:

\begin{Shaded}
\begin{Highlighting}[]
\NormalTok{$ }\KeywordTok{runhaskell} \NormalTok{Shakefile}
\end{Highlighting}
\end{Shaded}

And let the magic happen!

I run this enough times that I like to alias this:

\begin{Shaded}
\begin{Highlighting}[]
\CommentTok{# in ~/.bashrc}
\KeywordTok{alias} \NormalTok{shake=runhaskell Shakefile}
\end{Highlighting}
\end{Shaded}

Note that \texttt{want} specifies the \textbf{default} ``wants''. You can
specify your own collection by passing a parameter:

\begin{Shaded}
\begin{Highlighting}[]
\NormalTok{$ }\KeywordTok{runhaskell} \NormalTok{Shakefile out/report.doc}
\end{Highlighting}
\end{Shaded}

\section{Wildcards}\label{wildcards}

You may have noticed that even though we had multiple images in the \texttt{img}
folder, we required them all explicitly. This could cause problems. What if in
the future, our documents used more images?

We can define wildcards using \emph{shake}'s \texttt{getDirectoryFiles}, which
returns results of a wildcard search in an \texttt{Action} monad.
\texttt{getDirectoryFiles} takes a directory base and a list of wildcards.

\begin{Shaded}
\begin{Highlighting}[]
\CommentTok{-- Shakefile}

\OtherTok{srcFiles ::} \DataTypeTok{Action} \NormalTok{[FilePath]}
\NormalTok{srcFiles }\FunctionTok{=} \NormalTok{getDirectoryFiles }\StringTok{""}
    \NormalTok{[ }\StringTok{"src/report.md"}
    \NormalTok{, }\StringTok{"img/*.jpg"} \NormalTok{]}

\OtherTok{main ::} \DataTypeTok{IO} \NormalTok{()}
\NormalTok{main }\FunctionTok{=} \NormalTok{shakeArgs opts }\FunctionTok{$} \KeywordTok{do}
    \NormalTok{want [}\StringTok{"out/report.doc"}\NormalTok{,}\StringTok{"out/report.pdf"}\NormalTok{,}\StringTok{"out/report.html"}\NormalTok{]}

    \StringTok{"out/report.doc"} \FunctionTok{*>} \NormalTok{\textbackslash{}f }\OtherTok{->} \KeywordTok{do}
        \NormalTok{deps }\OtherTok{<-} \NormalTok{srcFiles}
        \NormalTok{need deps}
        \NormalTok{cmd }\StringTok{"pandoc"} \NormalTok{[ }\StringTok{"src/report.md"}\NormalTok{, }\StringTok{"-o"}\NormalTok{, f ]}

    \StringTok{"out/report.pdf"} \FunctionTok{*>} \NormalTok{\textbackslash{}f }\OtherTok{->} \KeywordTok{do}
        \NormalTok{deps }\OtherTok{<-} \NormalTok{srcFiles}
        \NormalTok{need deps}
        \NormalTok{cmd }\StringTok{"pandoc"} \NormalTok{[ }\StringTok{"src/report.md"}\NormalTok{, }\StringTok{"-o"}\NormalTok{, f, }\StringTok{"-V"}\NormalTok{, }\StringTok{"links-as-notes"} \NormalTok{]}

    \StringTok{"out/report.html"} \FunctionTok{*>} \NormalTok{\textbackslash{}f }\OtherTok{->} \KeywordTok{do}
        \NormalTok{deps }\OtherTok{<-} \NormalTok{srcFiles}
        \NormalTok{need }\FunctionTok{$} \StringTok{"css/report.css"} \FunctionTok{:} \NormalTok{deps}
        \NormalTok{cmd }\StringTok{"pandoc"} \NormalTok{[ }\StringTok{"src/report.md"}\NormalTok{, }\StringTok{"-o"}\NormalTok{, f, }\StringTok{"-c"}\NormalTok{, }\StringTok{"css/report.css"}\NormalTok{, }\StringTok{"-S"} \NormalTok{]}

    \StringTok{"img/img2.jpg"} \FunctionTok{*>} \NormalTok{\textbackslash{}f }\OtherTok{->} \KeywordTok{do}
        \NormalTok{cmd }\StringTok{"wget"} \NormalTok{[ }\StringTok{"http://example.com/img2.jpg"}\NormalTok{, }\StringTok{"-O"}\NormalTok{, f ]}

    \StringTok{"clean"} \FunctionTok{~>} \NormalTok{removeFilesAfter }\StringTok{".shake"} \NormalTok{[}\StringTok{"//*"}\NormalTok{]}
\end{Highlighting}
\end{Shaded}

If you are comfortable with applicative style, you can make it all happen on one
line:

\begin{Shaded}
\begin{Highlighting}[]
\StringTok{"out/report.doc"} \FunctionTok{*>} \NormalTok{\textbackslash{}f }\OtherTok{->} \KeywordTok{do}
    \NormalTok{need }\FunctionTok{<$>} \NormalTok{srcFiles}
\end{Highlighting}
\end{Shaded}

(You'll need to import \texttt{\textless{}\$\textgreater{}} from
\texttt{Control.Applicative}, and GHC will complain about the discarded value
unless you use \texttt{void} or enable \texttt{-fno-warn-wrong-do-bind})

\section{Phony Rules}\label{phony-rules}

Now, you might sometimes want rules that are ``just tasks'' that don't relate to
creating a specific file. That is, they still depend on other files or rules and
are triggered to update when their dependencies are out of date, but they just
aren't about building files.

For example, what if you wanted a task \texttt{build-some}, which builds only
\texttt{report.pdf} and \texttt{report.doc}, and outputs a proverb to the
command line?

One thing you can do is to simply use a rule with a name that does not
correspond to any file:

\begin{Shaded}
\begin{Highlighting}[]
\CommentTok{-- Bad}
\StringTok{"build-some"} \FunctionTok{*>} \NormalTok{\textbackslash{}_ }\OtherTok{->} \KeywordTok{do}
    \NormalTok{need [}\StringTok{"out/report.pdf"}\NormalTok{,}\StringTok{"out/report.doc"}\NormalTok{]}
    \NormalTok{cmd }\StringTok{"fortune"} \NormalTok{[}\StringTok{""}\NormalTok{]}
\end{Highlighting}
\end{Shaded}

However, this is kind of an inelegant solution. There really actually is not a
file \texttt{build-some}. Also, if someone ever decides to create a file called
\texttt{build-some}, you'll find that this rule never gets run.

The best way is to create a ``phony'' rule, which is a rule that is not tied to
a file. This is the reason for the alias I specified at the beginning of the
post:

\begin{Shaded}
\begin{Highlighting}[]
\CommentTok{-- Good}
\StringTok{"build-some"} \FunctionTok{~>} \KeywordTok{do}
    \NormalTok{need [}\StringTok{"out/report.pdf"}\NormalTok{,}\StringTok{"out/report.doc"}\NormalTok{]}
    \NormalTok{cmd }\StringTok{"fortune"} \NormalTok{[}\StringTok{""}\NormalTok{]}
\end{Highlighting}
\end{Shaded}

And voilà!

\subsection{Cleanup}\label{cleanup}

You might have noticed the phony rule in the scaffold Shakefile:

\begin{Shaded}
\begin{Highlighting}[]
\StringTok{"clean"} \FunctionTok{~>} \NormalTok{removeFilesAfter }\StringTok{".shake"} \NormalTok{[}\StringTok{"//*"}\NormalTok{]}
\end{Highlighting}
\end{Shaded}

If you run \texttt{shake\ clean}, it will remove all files in the
\texttt{.shake/} directory after the rule has completed its execution.
\texttt{removeFilesAfter} removes the files in the given base directory
(\texttt{.shake}) matching the given wildcards (\texttt{{[}"//*"{]}}) after all
rules have completed their course.

This is useful for cleaning up \emph{shake}'s metadata files after you are done
with your build, or if you want to run the task management system on a clean
start.

\section{Completed File}\label{completed-file}

\begin{Shaded}
\begin{Highlighting}[]
\CommentTok{-- Shakefile}
\OtherTok{\{-# OPTIONS_GHC -fno-warn-wrong-do-bind #-\}}

\KeywordTok{import }\DataTypeTok{Control.Applicative} \NormalTok{((<$>))}
\KeywordTok{import }\DataTypeTok{Development.Shake}

\NormalTok{opts }\FunctionTok{=} \NormalTok{shakeOptions \{ shakeFiles    }\FunctionTok{=} \StringTok{".shake/"} \NormalTok{\}}

\OtherTok{main ::} \DataTypeTok{IO} \NormalTok{()}
\NormalTok{main }\FunctionTok{=} \NormalTok{shakeArgs opts }\FunctionTok{$} \KeywordTok{do}
    \NormalTok{want [}\StringTok{"out/report.doc"}\NormalTok{,}\StringTok{"out/report.pdf"}\NormalTok{,}\StringTok{"out/report.html"}\NormalTok{]}

    \StringTok{"build-some"} \FunctionTok{~>} \KeywordTok{do}
        \NormalTok{need [}\StringTok{"out/report.pdf"}\NormalTok{,}\StringTok{"out/report.doc"}\NormalTok{]}
        \NormalTok{cmd }\StringTok{"fortune"} \NormalTok{[}\StringTok{""}\NormalTok{]}

    \StringTok{"out/report.doc"} \FunctionTok{*>} \NormalTok{\textbackslash{}f }\OtherTok{->} \KeywordTok{do}
        \NormalTok{need }\FunctionTok{<$>} \NormalTok{srcFiles}
        \NormalTok{cmd }\StringTok{"pandoc"} \NormalTok{[ }\StringTok{"src/report.md"}\NormalTok{, }\StringTok{"-o"}\NormalTok{, f ]}

    \StringTok{"out/report.pdf"} \FunctionTok{*>} \NormalTok{\textbackslash{}f }\OtherTok{->} \KeywordTok{do}
        \NormalTok{need }\FunctionTok{<$>} \NormalTok{srcFiles}
        \NormalTok{cmd }\StringTok{"pandoc"} \NormalTok{[ }\StringTok{"src/report.md"}\NormalTok{, }\StringTok{"-o"}\NormalTok{, f, }\StringTok{"-V"}\NormalTok{, }\StringTok{"links-as-notes"} \NormalTok{]}

    \StringTok{"out/report.html"} \FunctionTok{*>} \NormalTok{\textbackslash{}f }\OtherTok{->} \KeywordTok{do}
        \NormalTok{deps }\OtherTok{<-} \NormalTok{srcFiles}
        \NormalTok{need }\FunctionTok{$} \StringTok{"css/report.css"} \FunctionTok{:} \NormalTok{deps}
        \NormalTok{cmd }\StringTok{"pandoc"} \NormalTok{[ }\StringTok{"src/report.md"}\NormalTok{, }\StringTok{"-o"}\NormalTok{, f, }\StringTok{"-c"}\NormalTok{, }\StringTok{"css/report.css"}\NormalTok{, }\StringTok{"-S"} \NormalTok{]}

    \StringTok{"img/img2.jpg"} \FunctionTok{*>} \NormalTok{\textbackslash{}f }\OtherTok{->} \KeywordTok{do}
        \NormalTok{cmd }\StringTok{"wget"} \NormalTok{[ }\StringTok{"http://example.com/img2.jpg"}\NormalTok{, }\StringTok{"-O"}\NormalTok{, f ]}

    \StringTok{"clean"} \FunctionTok{~>} \NormalTok{removeFilesAfter }\StringTok{".shake"} \NormalTok{[}\StringTok{"//*"}\NormalTok{]}

\OtherTok{srcFiles ::} \DataTypeTok{Action} \NormalTok{[FilePath]}
\NormalTok{srcFiles }\FunctionTok{=} \NormalTok{getDirectoryFiles }\StringTok{""}
    \NormalTok{[ }\StringTok{"src/report.md"}
    \NormalTok{, }\StringTok{"img/*.jpg"} \NormalTok{]}
\end{Highlighting}
\end{Shaded}

\section{Wrapping Up}\label{wrapping-up}

If you look at the
\href{http://hackage.haskell.org/packages/archive/shake/0.10.6/doc/html/Development-Shake.html}{Shake
Documentation}, you will find a lot of ways you can build complex networks of
dependencies.

Hopefully there are enough use cases here to be useful in general applications.

\subsection{Monadic Tricks}\label{monadic-tricks}

Because everything is Haskell, you can easily generate rules using your basic
monad iterators by taking advantage of Haskell's extensive standard library of
monad functions. For example, if you want to generate multiple reports, you can
use \texttt{forM\_}:

\begin{Shaded}
\begin{Highlighting}[]
\KeywordTok{let} \NormalTok{reports }\FunctionTok{=} \NormalTok{[}\StringTok{"report1"}\NormalTok{, }\StringTok{"report2"}\NormalTok{, }\StringTok{"report3"}\NormalTok{]}

\NormalTok{want }\FunctionTok{$} \NormalTok{(\textbackslash{}s f }\OtherTok{->} \StringTok{"out/"} \FunctionTok{++} \NormalTok{s }\FunctionTok{++} \StringTok{"."} \FunctionTok{++} \NormalTok{f) }\FunctionTok{<$>}
    \NormalTok{[}\StringTok{"report1"}\NormalTok{,}\StringTok{"report2"}\NormalTok{,}\StringTok{"report3"}\NormalTok{] }\FunctionTok{<*>} \NormalTok{[}\StringTok{"doc"}\NormalTok{,}\StringTok{"pdf"}\NormalTok{,}\StringTok{"html"}\NormalTok{]}

\NormalTok{forM_ [}\StringTok{"report1"}\NormalTok{,}\StringTok{"report2"}\NormalTok{,}\StringTok{"report3"}\NormalTok{] }\FunctionTok{$} \NormalTok{\textbackslash{}reportName }\OtherTok{->} \KeywordTok{do}
    \KeywordTok{let}
        \NormalTok{outBase }\FunctionTok{=} \StringTok{"out/"} \FunctionTok{++} \NormalTok{reportName}
        \NormalTok{srcName }\FunctionTok{=} \StringTok{"src/"} \FunctionTok{++} \NormalTok{reportName }\FunctionTok{++} \StringTok{".md"}

    \NormalTok{outBase }\FunctionTok{++} \StringTok{".doc"} \FunctionTok{*>} \NormalTok{\textbackslash{}f }\OtherTok{->} \KeywordTok{do}
        \NormalTok{need }\FunctionTok{<$>} \NormalTok{srcFiles}
        \NormalTok{cmd }\StringTok{"pandoc"} \NormalTok{[ srcName, }\StringTok{"-o"}\NormalTok{, f ]}

    \NormalTok{outBase }\FunctionTok{++} \StringTok{".pdf"} \FunctionTok{*>} \NormalTok{\textbackslash{}f }\OtherTok{->} \KeywordTok{do}
        \NormalTok{need }\FunctionTok{<$>} \NormalTok{srcFiles}
        \NormalTok{cmd }\StringTok{"pandoc"} \NormalTok{[ srcName, }\StringTok{"-o"}\NormalTok{, f, }\StringTok{"-V"}\NormalTok{, }\StringTok{"links-as-notes"} \NormalTok{]}

    \NormalTok{outBase }\FunctionTok{++} \StringTok{".html"} \FunctionTok{*>} \NormalTok{\textbackslash{}f }\OtherTok{->} \KeywordTok{do}
        \NormalTok{deps }\OtherTok{<-} \NormalTok{srcFiles}
        \NormalTok{need }\FunctionTok{$} \StringTok{"css/report.css"} \FunctionTok{:} \NormalTok{deps}
        \NormalTok{cmd }\StringTok{"pandoc"} \NormalTok{[ srcName, }\StringTok{"-o"}\NormalTok{, f, }\StringTok{"-c"}\NormalTok{, }\StringTok{"css/report.css"}\NormalTok{, }\StringTok{"-S"} \NormalTok{]}
\end{Highlighting}
\end{Shaded}

Note however that you can get the same thing by just using wildcards (with
\texttt{takeFileName}). But this is just an example, feel free to let your
imagination roam!

\subsection{Looking Forward}\label{looking-forward}

We've seen how \emph{Shake} is good at setting up systems for managing and
executing dependencies. This is good for running simple system commands.
However, there is a lot more about scripting and task automation than managing
dependencies.

For example, almost everything we've done can be done with a simple Makefile.
What does Haskell offer to the scripting scene?

\subsubsection{Strong Typing}\label{strong-typing}

As you'll know, one of the magical things about Haskell is that because of its
expressive strong typing system, you leave the debugging to the compiler. If it
compiles, it works exactly the way you want!

This is pretty lacking in the bare-bones system we have in place now. Right now
we are just firing off arbitrary system commands that are basically specified in
strings with no type of typing. We can compile anything, whether there are bugs
in it or not.

Luckily \emph{Shake} is very good at integrating seamlessly with any kind of
framework. We can leave this up to other frameworks.

One popular framework for this that is gaining in maturity is
\href{http://hackage.haskell.org/package/shelly}{\emph{Shelly}} (A fork of an
older project that is an
\href{http://www.yesodweb.com/blog/2012/03/shelly-for-shell-scripts}{ongoing}
Yesod Project
\href{http://www.yesodweb.com/blog/2012/07/shelly-update}{experiment}), but you
are welcome to using your own. At the present Haskell is still developing and
growing in this aspect. I hope to eventually write an article about
\emph{Shelly} integration with \emph{Shake}.

\subsubsection{Other}\label{other}

These are just some ways to think about using \emph{Shake} in new more creative
ways. Let me know if you think of any clever integrations in the comments!

\end{document}
