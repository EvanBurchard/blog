\documentclass[]{article}
\usepackage{lmodern}
\usepackage{amssymb,amsmath}
\usepackage{ifxetex,ifluatex}
\usepackage{fixltx2e} % provides \textsubscript
\ifnum 0\ifxetex 1\fi\ifluatex 1\fi=0 % if pdftex
  \usepackage[T1]{fontenc}
  \usepackage[utf8]{inputenc}
\else % if luatex or xelatex
  \ifxetex
    \usepackage{mathspec}
    \usepackage{xltxtra,xunicode}
  \else
    \usepackage{fontspec}
  \fi
  \defaultfontfeatures{Mapping=tex-text,Scale=MatchLowercase}
  \newcommand{\euro}{€}
\fi
% use upquote if available, for straight quotes in verbatim environments
\IfFileExists{upquote.sty}{\usepackage{upquote}}{}
% use microtype if available
\IfFileExists{microtype.sty}{\usepackage{microtype}}{}
\usepackage[margin=1in]{geometry}
\usepackage{color}
\usepackage{fancyvrb}
\newcommand{\VerbBar}{|}
\newcommand{\VERB}{\Verb[commandchars=\\\{\}]}
\DefineVerbatimEnvironment{Highlighting}{Verbatim}{commandchars=\\\{\}}
% Add ',fontsize=\small' for more characters per line
\newenvironment{Shaded}{}{}
\newcommand{\KeywordTok}[1]{\textcolor[rgb]{0.00,0.44,0.13}{\textbf{{#1}}}}
\newcommand{\DataTypeTok}[1]{\textcolor[rgb]{0.56,0.13,0.00}{{#1}}}
\newcommand{\DecValTok}[1]{\textcolor[rgb]{0.25,0.63,0.44}{{#1}}}
\newcommand{\BaseNTok}[1]{\textcolor[rgb]{0.25,0.63,0.44}{{#1}}}
\newcommand{\FloatTok}[1]{\textcolor[rgb]{0.25,0.63,0.44}{{#1}}}
\newcommand{\ConstantTok}[1]{\textcolor[rgb]{0.53,0.00,0.00}{{#1}}}
\newcommand{\CharTok}[1]{\textcolor[rgb]{0.25,0.44,0.63}{{#1}}}
\newcommand{\SpecialCharTok}[1]{\textcolor[rgb]{0.25,0.44,0.63}{{#1}}}
\newcommand{\StringTok}[1]{\textcolor[rgb]{0.25,0.44,0.63}{{#1}}}
\newcommand{\VerbatimStringTok}[1]{\textcolor[rgb]{0.25,0.44,0.63}{{#1}}}
\newcommand{\SpecialStringTok}[1]{\textcolor[rgb]{0.73,0.40,0.53}{{#1}}}
\newcommand{\ImportTok}[1]{{#1}}
\newcommand{\CommentTok}[1]{\textcolor[rgb]{0.38,0.63,0.69}{\textit{{#1}}}}
\newcommand{\DocumentationTok}[1]{\textcolor[rgb]{0.73,0.13,0.13}{\textit{{#1}}}}
\newcommand{\AnnotationTok}[1]{\textcolor[rgb]{0.38,0.63,0.69}{\textbf{\textit{{#1}}}}}
\newcommand{\CommentVarTok}[1]{\textcolor[rgb]{0.38,0.63,0.69}{\textbf{\textit{{#1}}}}}
\newcommand{\OtherTok}[1]{\textcolor[rgb]{0.00,0.44,0.13}{{#1}}}
\newcommand{\FunctionTok}[1]{\textcolor[rgb]{0.02,0.16,0.49}{{#1}}}
\newcommand{\VariableTok}[1]{\textcolor[rgb]{0.10,0.09,0.49}{{#1}}}
\newcommand{\ControlFlowTok}[1]{\textcolor[rgb]{0.00,0.44,0.13}{\textbf{{#1}}}}
\newcommand{\OperatorTok}[1]{\textcolor[rgb]{0.40,0.40,0.40}{{#1}}}
\newcommand{\BuiltInTok}[1]{{#1}}
\newcommand{\ExtensionTok}[1]{{#1}}
\newcommand{\PreprocessorTok}[1]{\textcolor[rgb]{0.74,0.48,0.00}{{#1}}}
\newcommand{\AttributeTok}[1]{\textcolor[rgb]{0.49,0.56,0.16}{{#1}}}
\newcommand{\RegionMarkerTok}[1]{{#1}}
\newcommand{\InformationTok}[1]{\textcolor[rgb]{0.38,0.63,0.69}{\textbf{\textit{{#1}}}}}
\newcommand{\WarningTok}[1]{\textcolor[rgb]{0.38,0.63,0.69}{\textbf{\textit{{#1}}}}}
\newcommand{\AlertTok}[1]{\textcolor[rgb]{1.00,0.00,0.00}{\textbf{{#1}}}}
\newcommand{\ErrorTok}[1]{\textcolor[rgb]{1.00,0.00,0.00}{\textbf{{#1}}}}
\newcommand{\NormalTok}[1]{{#1}}
\ifxetex
  \usepackage[setpagesize=false, % page size defined by xetex
              unicode=false, % unicode breaks when used with xetex
              xetex]{hyperref}
\else
  \usepackage[unicode=true]{hyperref}
\fi
\hypersetup{breaklinks=true,
            bookmarks=true,
            pdfauthor={Justin Le},
            pdftitle={Intro to Machines \& Arrows (Part 1)},
            colorlinks=true,
            citecolor=blue,
            urlcolor=blue,
            linkcolor=magenta,
            pdfborder={0 0 0}}
\urlstyle{same}  % don't use monospace font for urls
% Make links footnotes instead of hotlinks:
\renewcommand{\href}[2]{#2\footnote{\url{#1}}}
\setlength{\parindent}{0pt}
\setlength{\parskip}{6pt plus 2pt minus 1pt}
\setlength{\emergencystretch}{3em}  % prevent overfull lines
\setcounter{secnumdepth}{0}

\title{Intro to Machines \& Arrows (Part 1)}
\author{Justin Le}

\begin{document}
\maketitle

\emph{Originally posted on
\textbf{\href{http://home.jle0.com:4111/entry/intro-to-machines-arrows-part-1-stream-and.html}{in
Code}}.}

So I'm going to be running a series soon on computation and (physical) simulations using AFRP
(Arrowized Functional Reactive Programming) principles.

I consider (A)FRP to actually be a pretty game changing paradigm. It provides us with semantics by
which to compose and build time-varying, reactive behaviors and completely changes the way we
approach any sort of simulation/state-like project.

However, it's fairly difficult to approach the subject (and not dumb things down to the point of
magic) without a solid foundation on the basic building blocks of AFRP: \textbf{machines}. And so
before I get started on my simulations in FRP series, here is hopefully a practical foundation on
the practical usage and construction of the machine design pattern that we can build on to get
started on our Physics simulation!

This post will assume a somewhat basic knowledge of Haskell. I'll try explaining concepts here and
there if I feel that they might not be very commonly known. But if you have any questions, feel free
to leave a comment or stop by freenode's \#haskell on irc!

In this post we are going to introduce Streams and (more importantly) Autos. In the next, we will be
looking at Autos as a member of the powerful Category and Arrow typeclasses. Finally, we will bring
in the final machine, the Wire, to transition into the popular AFRP library
\href{http://hackage.haskell.org/package/netwire}{netwire}.

\section{Why FRP?}\label{why-frp}

So why do we even bother with FRP? Why not just wrap everything in a giant global state monad and
program imperatively?

To get to the bottom of this, we must remember why we even bother with functional programming in the
first place. Think very hard about this question, because without a real answer, all of this will be
a waste. Is it just because it's cool?

The (a?) reason why (pure) functional programming is powerful is because is a subset of what can be
called
\textbf{\href{http://www.haskellforall.com/2012/08/the-category-design-pattern.html}{compositional
programming}}. Almost all of the benefits of functional programming can be drawn from its unmatched
power of composability. Think of the power of unix pipes. We can build programs by the free
composition of smaller, simpler self-contained programs and concepts that don't have to be aware of
any other part. Recognizing this completely changes the way we approach problems.

So why FRP? FRP provides for us meaningful semantics by which to \emph{compose} time-varying and
reactive behaviors, and create complex ones from smaller ones. You reason about a self-contained
behavior ``wire''\ldots{}and then use tools to build complex behaviors from simple ones. You don't
specify what happens when, you specify how things \emph{should behave}, from a top-down level. And
then compose those behaviors.

Yadda yadda. I am probably boring you at this point, let's just get started and jump into machines!

\section{Streams}\label{streams}

Let's start with streams, one of the simpler of machines.

Streams are basically infinitely long linked lists.

\begin{Shaded}
\begin{Highlighting}[]
\KeywordTok{data} \DataTypeTok{Stream} \NormalTok{b }\FunctionTok{=} \DataTypeTok{SCons} \NormalTok{(b, }\DataTypeTok{Stream} \NormalTok{b)}
\end{Highlighting}
\end{Shaded}

(\texttt{SCons} for ``Stream cons'')

Compare that with the linked list data type, which is a Stream with an Ending (\texttt{Nil}):

\begin{Shaded}
\begin{Highlighting}[]
\KeywordTok{data} \DataTypeTok{List} \NormalTok{a }\FunctionTok{=} \DataTypeTok{Cons} \NormalTok{(a, }\DataTypeTok{List} \NormalTok{a) }\FunctionTok{|} \DataTypeTok{Nil}
  \KeywordTok{deriving} \NormalTok{(}\DataTypeTok{Show}\NormalTok{)}
\end{Highlighting}
\end{Shaded}

or, as is more traditionally written:

\begin{Shaded}
\begin{Highlighting}[]
\KeywordTok{data} \NormalTok{[a] }\FunctionTok{=} \NormalTok{(}\FunctionTok{:}\NormalTok{) a [a] }\FunctionTok{|} \NormalTok{[]}
\end{Highlighting}
\end{Shaded}

It's pretty easy to build lists:

\begin{Shaded}
\begin{Highlighting}[]
\OtherTok{myList ::} \DataTypeTok{List} \DataTypeTok{Int}
\NormalTok{myList }\FunctionTok{=} \DataTypeTok{Cons} \NormalTok{( }\DecValTok{1}\NormalTok{, }\DataTypeTok{Cons} \NormalTok{( }\DecValTok{2}\NormalTok{, }\DataTypeTok{Cons} \NormalTok{(}\DecValTok{3}\NormalTok{, }\DataTypeTok{Nil}\NormalTok{) ) )}
\end{Highlighting}
\end{Shaded}

which is just, in the more traditional (infix) form:

\begin{Shaded}
\begin{Highlighting}[]
\OtherTok{myList' ::} \NormalTok{[}\DataTypeTok{Int}\NormalTok{]}
\NormalTok{myList' }\FunctionTok{=} \DecValTok{1}\FunctionTok{:}\NormalTok{(}\DecValTok{2}\FunctionTok{:}\NormalTok{(}\DecValTok{3}\FunctionTok{:}\NormalTok{[]))}
\end{Highlighting}
\end{Shaded}

Let's see if \texttt{myList} does what we want: (a list from 1 to 3):

\begin{Shaded}
\begin{Highlighting}[]
\NormalTok{λ}\FunctionTok{:} \KeywordTok{let} \NormalTok{(}\DataTypeTok{Cons} \NormalTok{(x,xs)) }\FunctionTok{=} \NormalTok{myList}
\NormalTok{λ}\FunctionTok{:} \NormalTok{x}
\DecValTok{1}
\NormalTok{λ}\FunctionTok{:} \FunctionTok{:}\NormalTok{t xs}
\OtherTok{xs ::} \DataTypeTok{List} \DataTypeTok{Int}
\NormalTok{λ}\FunctionTok{:} \KeywordTok{let} \NormalTok{(}\DataTypeTok{Cons} \NormalTok{(y,ys)) }\FunctionTok{=} \NormalTok{xs}
\NormalTok{λ}\FunctionTok{:} \NormalTok{y}
\DecValTok{2}
\NormalTok{λ}\FunctionTok{:} \KeywordTok{let} \NormalTok{(}\DataTypeTok{Cons} \NormalTok{(z,zs)) }\FunctionTok{=} \NormalTok{ys}
\NormalTok{λ}\FunctionTok{:} \NormalTok{z}
\DecValTok{3}
\NormalTok{λ}\FunctionTok{:} \NormalTok{zs}
\DataTypeTok{Nil}
\end{Highlighting}
\end{Shaded}

Yes! Perfect. We can ``traverse'' down our linked list by repeatedly pattern matching out the
``head'' (the \texttt{x}, the first part of the tuple) and the ``tail'' (the \texttt{xs}, the second
part of the tuple).

But how are we going to build a stream\ldots{}? We don't have access to \texttt{Nil}, so do we have
to manually type out an infinite stream?

Let's try defining the stream \texttt{{[}1..{]}} --- a stream that contains every natural number
starting from 1.

\begin{Shaded}
\begin{Highlighting}[]
\OtherTok{myStream' ::} \DataTypeTok{Stream} \DataTypeTok{Int}
\NormalTok{myStream' }\FunctionTok{=} \DataTypeTok{SCons} \NormalTok{( }\DecValTok{1}\NormalTok{, }\DataTypeTok{SCons} \NormalTok{( }\DecValTok{2}\NormalTok{, }\DataTypeTok{SCons} \NormalTok{( }\DecValTok{3}\NormalTok{, }\FunctionTok{...} \NormalTok{) ) )}
\end{Highlighting}
\end{Shaded}

Hm. This is going to take a while. I wonder if there's an easier way.

We can take advantage of Haskell's ``lazy-by-default''-ness and leave the ``rest'' of the stream as
an unevaluated function call. And then we can recurse!

\begin{Shaded}
\begin{Highlighting}[]
\OtherTok{myStream ::} \DataTypeTok{Stream} \DataTypeTok{Int}
\NormalTok{myStream }\FunctionTok{=} \NormalTok{streamFrom }\DecValTok{1}
  \KeywordTok{where}
\OtherTok{    streamFrom ::} \DataTypeTok{Int} \OtherTok{->} \DataTypeTok{Stream} \DataTypeTok{Int}
    \NormalTok{streamFrom n }\FunctionTok{=} \DataTypeTok{SCons} \NormalTok{( n, streamFrom (n}\FunctionTok{+}\DecValTok{1}\NormalTok{) )}
\end{Highlighting}
\end{Shaded}

Cool! Let's see if this \texttt{myStream} really does what we want, the same way we tested
\texttt{myList}:

\begin{Shaded}
\begin{Highlighting}[]
\NormalTok{λ}\FunctionTok{:} \KeywordTok{let} \NormalTok{(}\DataTypeTok{SCons} \NormalTok{(x, xs)) }\FunctionTok{=} \NormalTok{myStream}
\NormalTok{λ}\FunctionTok{:} \NormalTok{x}
\DecValTok{1}
\NormalTok{λ}\FunctionTok{:} \FunctionTok{:}\NormalTok{t xs}
\OtherTok{xs ::} \DataTypeTok{Stream} \DataTypeTok{Int}
\NormalTok{λ}\FunctionTok{:} \KeywordTok{let} \NormalTok{(}\DataTypeTok{SCons} \NormalTok{(y, ys)) }\FunctionTok{=} \NormalTok{xs}
\NormalTok{λ}\FunctionTok{:} \NormalTok{y}
\DecValTok{2}
\NormalTok{λ}\FunctionTok{:} \KeywordTok{let} \NormalTok{(}\DataTypeTok{SCons} \NormalTok{(z, zs)) }\FunctionTok{=} \NormalTok{ys}
\NormalTok{λ}\FunctionTok{:} \NormalTok{z}
\DecValTok{3}
\NormalTok{λ}\FunctionTok{:} \KeywordTok{let} \NormalTok{(}\DataTypeTok{SCons} \NormalTok{(j,js)) }\FunctionTok{=} \NormalTok{zs}
\NormalTok{λ}\FunctionTok{:} \NormalTok{j}
\DecValTok{4}
\NormalTok{λ}\FunctionTok{:} \KeywordTok{let} \NormalTok{(}\DataTypeTok{SCons} \NormalTok{(k,ks)) }\FunctionTok{=} \NormalTok{js}
\NormalTok{λ}\FunctionTok{:} \NormalTok{k}
\DecValTok{5}
\end{Highlighting}
\end{Shaded}

Yes, it works perfectly! Just like in the case of List, we can ``traverse'' down the stream by
pattern matching out the ``head'' of the stream (the first part of the tuple) and the ``tail'' of
the stream (the second part of the tuple).

Note that we can use some syntax trickery with our data definition of \texttt{Stream} by adding a
label to the first (and only) field:

\begin{Shaded}
\begin{Highlighting}[]
\KeywordTok{data} \DataTypeTok{Stream} \NormalTok{b }\FunctionTok{=} \DataTypeTok{SCons} \NormalTok{\{}\OtherTok{ runStream ::} \NormalTok{(b, }\DataTypeTok{Stream} \NormalTok{b) \}}
\end{Highlighting}
\end{Shaded}

and we can do fancy things like:

\begin{Shaded}
\begin{Highlighting}[]
\NormalTok{λ}\FunctionTok{:} \FunctionTok{:}\NormalTok{t runStream}
\OtherTok{runStream ::} \DataTypeTok{Stream} \NormalTok{b }\OtherTok{->} \NormalTok{(b, }\DataTypeTok{Stream} \NormalTok{b)}
\NormalTok{λ}\FunctionTok{:} \KeywordTok{let} \NormalTok{(x, xs) }\FunctionTok{=} \NormalTok{runStream myStream}
\NormalTok{λ}\FunctionTok{:} \NormalTok{x}
\DecValTok{1}
\end{Highlighting}
\end{Shaded}

Basically, we get for free the function \texttt{runStream}, a function that yanks the tuple out of
the stream.

One minor final touch --- because \texttt{Stream} has only one constructor and one field, we can
make it a \texttt{newtype}, which has similar usage patterns/syntax as a \texttt{data}, but which
the compiler can more easily optimize:

\begin{Shaded}
\begin{Highlighting}[]
\KeywordTok{newtype} \DataTypeTok{Stream} \NormalTok{b }\FunctionTok{=} \DataTypeTok{SCons} \NormalTok{\{}\OtherTok{ runStream ::} \NormalTok{(b, }\DataTypeTok{Stream} \NormalTok{b) \}}
\end{Highlighting}
\end{Shaded}

\subsubsection{Automating Traversal}\label{automating-traversal}

The repeated pattern matching is actually kind of tedious, and it'll only get more annoying over
time, so let's make a function that can automate the pattern matching for us really quickly so that
we can test it more easily.

\texttt{streamToList} will take a Stream and perform the very straightforward conversion into an
infinite list.

\begin{Shaded}
\begin{Highlighting}[]
\OtherTok{streamToList ::} \DataTypeTok{Stream} \NormalTok{b }\OtherTok{->} \NormalTok{[b]}
\NormalTok{streamToList (}\DataTypeTok{SCons} \NormalTok{(x, xs)) }\FunctionTok{=} \NormalTok{x }\FunctionTok{:} \NormalTok{streamToList xs}
\end{Highlighting}
\end{Shaded}

So now we can do:

\begin{Shaded}
\begin{Highlighting}[]
\NormalTok{λ}\FunctionTok{:} \NormalTok{take }\DecValTok{10} \FunctionTok{$} \NormalTok{streamToList myStream}
\NormalTok{[}\DecValTok{1}\NormalTok{,}\DecValTok{2}\NormalTok{,}\DecValTok{3}\NormalTok{,}\DecValTok{4}\NormalTok{,}\DecValTok{5}\NormalTok{,}\DecValTok{6}\NormalTok{,}\DecValTok{7}\NormalTok{,}\DecValTok{8}\NormalTok{,}\DecValTok{9}\NormalTok{,}\DecValTok{10}\NormalTok{]}
\end{Highlighting}
\end{Shaded}

Alternatively (and for reasons which will later become clear), we can also define
\texttt{testStream}, which takes a specified amount of elements and returns also the ``resulting''
stream after all of those steps, and \texttt{testStream\_}, which is the same thing except that we
throw away the modified stream.

\begin{Shaded}
\begin{Highlighting}[]
\OtherTok{testStream ::} \DataTypeTok{Stream} \NormalTok{b }\OtherTok{->} \DataTypeTok{Int} \OtherTok{->} \NormalTok{([b], }\DataTypeTok{Stream} \NormalTok{b)}
\NormalTok{testStream strm }\DecValTok{0} \FunctionTok{=} \NormalTok{([]  , strm )}
\NormalTok{testStream strm n }\FunctionTok{=} \NormalTok{(y}\FunctionTok{:}\NormalTok{ys, final)}
  \KeywordTok{where}
    \NormalTok{(y , next )   }\FunctionTok{=} \NormalTok{runStream  strm}
    \NormalTok{(ys, final)   }\FunctionTok{=} \NormalTok{testStream next (n}\FunctionTok{-}\DecValTok{1}\NormalTok{)}

\OtherTok{testStream_ ::} \DataTypeTok{Stream} \NormalTok{b }\OtherTok{->} \DataTypeTok{Int} \OtherTok{->} \NormalTok{[b]}
\NormalTok{testStream_ }\FunctionTok{=} \NormalTok{(fst }\FunctionTok{.}\NormalTok{) }\FunctionTok{.} \NormalTok{testStream}
\end{Highlighting}
\end{Shaded}

\begin{Shaded}
\begin{Highlighting}[]
\NormalTok{λ}\FunctionTok{:} \NormalTok{testStream_ myStream }\DecValTok{10}
\NormalTok{[}\DecValTok{1}\NormalTok{,}\DecValTok{2}\NormalTok{,}\DecValTok{3}\NormalTok{,}\DecValTok{4}\NormalTok{,}\DecValTok{5}\NormalTok{,}\DecValTok{6}\NormalTok{,}\DecValTok{7}\NormalTok{,}\DecValTok{8}\NormalTok{,}\DecValTok{9}\NormalTok{,}\DecValTok{10}\NormalTok{]}
\end{Highlighting}
\end{Shaded}

\subsection{Streams are nice}\label{streams-are-nice}

Streams are nice! If you've been using Haskell for any stretch of time, you'll know that we use
infinite lists all the time and to great usefulness for the ends of expressiveness and abstraction.

In math, streams are known as a form of \href{http://en.wikipedia.org/wiki/Moore_machine}{Moore
machines} (albeit with potentially infinite sets of state and output values). They are machines that
basically progress from state to state to state to state. They just keep on marching on\ldots{}like
a machine. In \texttt{myStream}, the initial state is 1. The next state is 2; the next is 3, etc.
What is important is that the next state \emph{is a function of the current state}.

This is made very apparent in our definition of \texttt{streamFrom}:

\begin{Shaded}
\begin{Highlighting}[]
\OtherTok{streamFrom ::} \DataTypeTok{Int} \OtherTok{->} \DataTypeTok{Stream} \DataTypeTok{Int}
\NormalTok{streamFrom n }\FunctionTok{=} \DataTypeTok{SCons} \NormalTok{( n, streamFrom (n}\FunctionTok{+}\DecValTok{1}\NormalTok{) )}
\end{Highlighting}
\end{Shaded}

The ``current state'' whenever we call \texttt{streamFrom\ n} is \texttt{n}\ldots{}the ``next
state'' (the ``initial state'' of the ``tail'') is \texttt{n+1}. We could have provided any function
\texttt{::\ Int\ -\textgreater{}\ Int} there (say, \texttt{n+2}), that would be for us our ``next
state'' function.

So \texttt{myStream} is a Moore-like machine whose ``next state'' function is ``the current state
plus one''.

\subsubsection{on State}\label{on-state}

Before we move on, let's take a slight diversion --- you can skip this whole section if you wish, to
get onto the main point.

Notice that in our last example, our ``state'' \texttt{n} was the same type as our ``output''
\texttt{n+1}. Is this in general the case?

The type of our stream is \texttt{Stream\ Int}\ldots{}does \texttt{Int} refer to the state or the
output?

As it turns out, this \texttt{Int} refers to the output --- it is the type of the ``head'' of the
stream, and all values we will eventually grab from it.

Note however that the type of the state is not actually included in the type signature, so it can be
whatever you want! As a trivial example, let's say that \texttt{streamFrom} actually takes a
\texttt{x\ ::\ Double} instead of an \texttt{n\ ::\ Int}, and rounds it before it pops it out as the
``head'':

\begin{Shaded}
\begin{Highlighting}[]
\OtherTok{myStream' ::} \DataTypeTok{Stream} \DataTypeTok{Int}
\NormalTok{myStream' }\FunctionTok{=} \NormalTok{streamFrom' }\FloatTok{1.0}
  \KeywordTok{where}
\OtherTok{    streamFrom' ::} \DataTypeTok{Double} \OtherTok{->} \DataTypeTok{Stream} \DataTypeTok{Int}
    \NormalTok{streamFrom' x }\FunctionTok{=} \DataTypeTok{SCons} \NormalTok{( round x, streamFrom' (x}\FunctionTok{+}\DecValTok{1}\NormalTok{) )}
\end{Highlighting}
\end{Shaded}

This function now sorta behaves similarly to our original \texttt{streamFrom}\ldots{}
\texttt{myStream\textquotesingle{}} would still be \texttt{Stream\ Int} and output ints, and we
might not ever be able to tell that the internal state of \texttt{myStream\textquotesingle{}} was
``actually'' a double!

Now also observe that the internal state is actually in general \emph{inaccessible} and
\emph{opaque} to the outside world. What if we had \texttt{streamFrom} simply return whether or not
\texttt{n} was even?

\begin{Shaded}
\begin{Highlighting}[]
\OtherTok{myBoolStream ::} \DataTypeTok{Stream} \DataTypeTok{Bool}
\NormalTok{myBoolStream }\FunctionTok{=} \NormalTok{boolStreamFrom }\DecValTok{1}
  \KeywordTok{where}
\OtherTok{    boolStreamFrom ::} \DataTypeTok{Int} \OtherTok{->} \DataTypeTok{Stream} \DataTypeTok{Bool}
    \NormalTok{boolStreamFrom n }\FunctionTok{=} \DataTypeTok{SCons} \NormalTok{( even n, boolStreamFrom (n}\FunctionTok{+}\DecValTok{1}\NormalTok{) )}
\end{Highlighting}
\end{Shaded}

\begin{Shaded}
\begin{Highlighting}[]
\NormalTok{λ}\FunctionTok{:} \NormalTok{take }\DecValTok{5} \FunctionTok{$} \NormalTok{streamToList myBoolStream}
\NormalTok{[}\DataTypeTok{False}\NormalTok{,}\DataTypeTok{True}\NormalTok{,}\DataTypeTok{False}\NormalTok{,}\DataTypeTok{True}\NormalTok{,}\DataTypeTok{False}\NormalTok{]}
\end{Highlighting}
\end{Shaded}

\texttt{myBoolStream} simply cycles between \texttt{False} and \texttt{True}. Yet, it has an
internal state that is completely closed off to us that is an \texttt{Int} counting from \texttt{1}
to infinity. We might not have ever even known.

This property --- that the states of these types of machines are hidden from the world --- is
actually going to be very useful. Like I said before, every machine can really be considered
self-contained. This is unlike using a State monad-based loop, where all internal state is more or
less freely viewable manipulatable by anyone. Here, every machine is truly its own little world.

In fact, because the type of the state is unknown and unpredictable\ldots{}even if we could
``force'' the state out of a stream somehow, we wouldn't even be able to work with it in a type safe
way. The type is truly dynamic and the type of the nth state of a stream is unknowable at compile
time.

Here is a stream whose state switches from an \texttt{Int} to a \texttt{Bool} dynamically.

\begin{verbatim}
wackyStateStream :: Stream (Maybe Int)
wackyStateStream = wackyStateBool True
  where
    wackyStateBool :: Bool -> Stream (Maybe Int)
    wakcyStateBool False  = SCons (Nothing , wackyStateBool True)
    wackyStateBool True   = SCons (Just 100, wackyStateInt 8)

    wackyStateInt :: Int -> Stream (Maybe Int)
    wackyStateInt n
        | n `mod` 7 == 0  = SCons (Just n, wackyStateBool True)
        | otherwise       = SCons (Just (n+2), wackyStateInt (n+3))

λ: take 7 $ streamToList wackyStateStream
[Nothing, Just 100, Just 8, Just 11, Just 16, Nothing, Just 100]
\end{verbatim}

\subsection{Continuing on}\label{continuing-on}

The problem with streams, as you might have guessed, is that you can't really affect their progress
once they start. Once you start \texttt{myStream}, it'll keep on marching on, and on, and
on\ldots{}you have no way to ``influence'' its progression \emph{during} its march. The
\emph{behavior} of our stream \emph{can't be influenced} by the outside world in any way, once it
has started. This is a bit limiting, because we want behaviors that we can have interact with each
other.

And so, we have the natural generalization of streams (and the machine we will be spending the most
time looking at): Auto.

\section{Auto}\label{auto}

Let's upgrade our streams, and introduce a way to affect how they progress. Let's call it an Auto.

\begin{Shaded}
\begin{Highlighting}[]
\KeywordTok{newtype} \DataTypeTok{Auto} \NormalTok{a b }\FunctionTok{=} \DataTypeTok{ACons} \NormalTok{\{}\OtherTok{ runAuto ::} \NormalTok{a }\OtherTok{->} \NormalTok{(b, }\DataTypeTok{Auto} \NormalTok{a b) \}}
\end{Highlighting}
\end{Shaded}

Now, instead of an \texttt{SCons} containing just a tuple (a head-tails), an \texttt{ACons} contains
a \emph{function} that \emph{produces} your head-tails tuple. Before, all of our \texttt{runStreams}
produced the same tuple no matter what. Now, our \texttt{runAuto} can produce a different tuple
based on an outside input.

This is cool!

Let's look at the type signature of Auto before we go too much further.

In \texttt{Auto\ a\ b}, \texttt{b} is the type of your ``head'' and the type of the items in your
``tail''. It's the type of your ``stream''.

\texttt{a} is the type of the ``influencing input''.

So now, we basically have a \texttt{Stream\ b}, except at every ``step'', we can ``influence'' the
Stream with something of type \texttt{a}.

\subsection{A Trivial Auto}\label{a-trivial-auto}

Let's look at a direct ``port'' of our \texttt{myStream}:

\begin{Shaded}
\begin{Highlighting}[]
\OtherTok{myStreamAuto ::} \DataTypeTok{Auto} \NormalTok{a }\DataTypeTok{Int}
\NormalTok{myStreamAuto }\FunctionTok{=} \NormalTok{streamAutoFrom }\DecValTok{1}
  \KeywordTok{where}
\OtherTok{    streamAutoFrom ::} \DataTypeTok{Int} \OtherTok{->} \DataTypeTok{Auto} \NormalTok{a }\DataTypeTok{Int}
    \NormalTok{streamAutoFrom n }\FunctionTok{=} \DataTypeTok{ACons} \FunctionTok{$} \NormalTok{\textbackslash{}_ }\OtherTok{->} \NormalTok{( n, streamAutoFrom (n}\FunctionTok{+}\DecValTok{1}\NormalTok{) )}
\end{Highlighting}
\end{Shaded}

This is kind of a dumb example, but \texttt{myStreamAuto} is just the exact same as
\texttt{myStream}. It's an Auto, but it \emph{ignores its influencing input}.

Let's try it out.

\begin{Shaded}
\begin{Highlighting}[]
\NormalTok{λ}\FunctionTok{:} \FunctionTok{:}\NormalTok{t runAuto}
\OtherTok{runAuto ::} \DataTypeTok{Auto} \NormalTok{a b }\OtherTok{->} \NormalTok{a }\OtherTok{->} \NormalTok{(b, }\DataTypeTok{Auto} \NormalTok{a b)}
\NormalTok{λ}\FunctionTok{:} \KeywordTok{let} \NormalTok{(x, xs) }\FunctionTok{=} \NormalTok{runAuto myStreamAuto undefined}
\NormalTok{λ}\FunctionTok{:} \NormalTok{x}
\DecValTok{1}
\NormalTok{λ}\FunctionTok{:} \FunctionTok{:}\NormalTok{t xs}
\OtherTok{xs ::} \DataTypeTok{Auto} \NormalTok{a }\DataTypeTok{Int}
\NormalTok{λ}\FunctionTok{:} \KeywordTok{let} \NormalTok{(y, ys) }\FunctionTok{=} \NormalTok{runAuto xs undefined}
\NormalTok{λ}\FunctionTok{:} \NormalTok{y}
\DecValTok{2}
\NormalTok{λ}\FunctionTok{:} \KeywordTok{let} \NormalTok{(z, zs) }\FunctionTok{=} \NormalTok{runAuto ys undefined}
\NormalTok{λ}\FunctionTok{:} \NormalTok{z}
\DecValTok{3}
\end{Highlighting}
\end{Shaded}

Remember that we are really doing \texttt{(runAuto\ myStreamAuto)\ undefined}, but because of how
Haskell associates function calls, the parentheses are unnecessary. And hey, it kind of looks like
\texttt{runAuto} is a two-parameter function with an Auto as the first parameter and the
``influence''/``input'' as its second. Which, due to the magic of currying-by-default, it really is!

\subsection{A Non-trivial Auto}\label{a-non-trivial-auto}

Okay, that was fun I guess. But now let's take a first look at an auto which ``can'' be influenced.

Let's have a \emph{resettable counter}. Kind of like \texttt{myStreamAuto}, but at every step, you
can choose to ``reset'' the count back to zero.

Actually, let's just jump to something even bigger. At every step, we can choose to ``set'' the
counter to whatever int we like.

We can do this by having the influence/input be a \texttt{Maybe\ Int}. If we want the counter to
progress normally, we pass in a \texttt{Nothing}. If we want the counter to reset to a number
\texttt{n} of our choosing, we pass in a \texttt{Just\ n}

\begin{Shaded}
\begin{Highlighting}[]
\OtherTok{settableAuto ::} \DataTypeTok{Auto} \NormalTok{(}\DataTypeTok{Maybe} \DataTypeTok{Int}\NormalTok{) }\DataTypeTok{Int}
\NormalTok{settableAuto }\FunctionTok{=} \NormalTok{settableCounterFrom }\DecValTok{1}
  \KeywordTok{where}
\OtherTok{    settableCounterFrom ::} \DataTypeTok{Int} \OtherTok{->} \DataTypeTok{Auto} \NormalTok{(}\DataTypeTok{Maybe} \DataTypeTok{Int}\NormalTok{) }\DataTypeTok{Int}
    \NormalTok{settableCounterFrom n }\FunctionTok{=} \DataTypeTok{ACons} \FunctionTok{$} \NormalTok{\textbackslash{}reset }\OtherTok{->}
      \KeywordTok{let} \NormalTok{c }\FunctionTok{=} \NormalTok{fromMaybe n reset}
      \KeywordTok{in}  \NormalTok{( c, settableCounterFrom (c }\FunctionTok{+} \DecValTok{1}\NormalTok{) )}
\end{Highlighting}
\end{Shaded}

Remember that \texttt{fromMaybe\ ::\ a\ -\textgreater{}\ Maybe\ a\ -\textgreater{}\ a} takes a
``default'' value, a Maybe value, and then returns the value inside the Maybe if it's a
\texttt{Just}, or the default value if it's a \texttt{Nothing}.

So basically, when you \texttt{runAuto} with the Auto, if you give it a \texttt{Nothing}, it'll give
you \texttt{(\ n,\ settableCounterFrom\ (n+1)\ )}. If you give it \texttt{Just\ m}, it'll give you
\texttt{(\ m,\ settableCounterFrom\ (m+1)\ )}.

Cool, let's try it out.

\begin{Shaded}
\begin{Highlighting}[]
\NormalTok{λ}\FunctionTok{:} \KeywordTok{let} \NormalTok{(x, xs) }\FunctionTok{=} \NormalTok{runAuto settableAuto }\DataTypeTok{Nothing}
\NormalTok{λ}\FunctionTok{:} \NormalTok{x}
\DecValTok{1}
\NormalTok{λ}\FunctionTok{:} \KeywordTok{let} \NormalTok{(y, ys) }\FunctionTok{=} \NormalTok{runAuto xs }\DataTypeTok{Nothing}
\NormalTok{λ}\FunctionTok{:} \NormalTok{y}
\DecValTok{2}
\NormalTok{λ}\FunctionTok{:} \KeywordTok{let} \NormalTok{(z, zs) }\FunctionTok{=} \NormalTok{runAuto ys (}\DataTypeTok{Just} \DecValTok{10}\NormalTok{)}
\NormalTok{λ}\FunctionTok{:} \NormalTok{z}
\DecValTok{10}
\NormalTok{λ}\FunctionTok{:} \KeywordTok{let} \NormalTok{(j, js) }\FunctionTok{=} \NormalTok{runAuto zs }\DataTypeTok{Nothing}
\NormalTok{λ}\FunctionTok{:} \NormalTok{j}
\DecValTok{11}
\NormalTok{λ}\FunctionTok{:} \KeywordTok{let} \NormalTok{(k, ks) }\FunctionTok{=} \NormalTok{runAuto js }\DataTypeTok{Nothing}
\NormalTok{λ}\FunctionTok{:} \NormalTok{k}
\DecValTok{12}
\NormalTok{λ}\FunctionTok{:} \KeywordTok{let} \NormalTok{(l, ls) }\FunctionTok{=} \NormalTok{runAuto ks (}\DataTypeTok{Just} \NormalTok{(}\FunctionTok{-}\DecValTok{1}\NormalTok{))}
\NormalTok{λ}\FunctionTok{:} \NormalTok{l}
\FunctionTok{-}\DecValTok{1}
\NormalTok{λ}\FunctionTok{:} \KeywordTok{let} \NormalTok{(m, ms) }\FunctionTok{=} \NormalTok{runAuto ls }\DataTypeTok{Nothing}
\NormalTok{λ}\FunctionTok{:} \NormalTok{m}
\DecValTok{0}
\end{Highlighting}
\end{Shaded}

And there ya go.

\subsubsection{Automatic traversals for our Autos}\label{automatic-traversals-for-our-autos}

Again, the manual pattern matching is a little tedious so let's write us a function to automate
``progressing'' down an Auto.

Like our \texttt{testStream}, \texttt{testAuto} takes an Auto. But because every ``step'' needs an
input, \texttt{testAuto\ auto} takes a \emph{list} that specifies the input for every step.
\texttt{testAuto} returns the resulting collection of results, and also the modified Auto.
\texttt{testAuto\_} throws away the new Auto and just gives us the collection.

\begin{Shaded}
\begin{Highlighting}[]
\OtherTok{testAuto ::} \DataTypeTok{Auto} \NormalTok{a b }\OtherTok{->} \NormalTok{[a] }\OtherTok{->} \NormalTok{([b], }\DataTypeTok{Auto} \NormalTok{a b)}
\NormalTok{testAuto auto []      }\FunctionTok{=} \NormalTok{([]  , auto )}
\NormalTok{testAuto auto (x}\FunctionTok{:}\NormalTok{xs)  }\FunctionTok{=} \NormalTok{(y}\FunctionTok{:}\NormalTok{ys, final)}
  \KeywordTok{where}
    \NormalTok{(y,  next ) }\FunctionTok{=} \NormalTok{runAuto  auto x}
    \NormalTok{(ys, final) }\FunctionTok{=} \NormalTok{testAuto next xs}

\OtherTok{testAuto_ ::} \DataTypeTok{Auto} \NormalTok{a b }\OtherTok{->} \NormalTok{[a] }\OtherTok{->} \NormalTok{[b]}
\NormalTok{testAuto_ }\FunctionTok{=} \NormalTok{(fst }\FunctionTok{.}\NormalTok{) }\FunctionTok{.} \NormalTok{testAuto}
\end{Highlighting}
\end{Shaded}

Trying it out on \texttt{settableAuto}:

\begin{verbatim}
λ: testAuto settableAuto [ Nothing, Nothing, Just 10
                         , Nothing, Nothing, Just (-1)
                         , Nothing ]
[1,2,10,11,12,-1,0]
\end{verbatim}

\subsection{A Shift}\label{a-shift}

Let's shift our thinking a bit. Instead of seeing Autos as ``streams you can influence'', we can
think about them as ``functions with state''. That is, they are functions that carry self-contained
encapsulated state \emph{inside} themselves. They are still \emph{functions} of some kind --- you
put in an ``input'' of type \texttt{a} and get in an ``output'' (the head) of type \texttt{b}.
However, every ``time'' you do this, you get a different output depending on what you have already
passed in and what the internal state is. As we will see, this internal state is completely opaque
to the world. The world only has access to the ``output'', the result.

(Remember that, because we're in a functional language, nothing is technically actually really
``mutable''. When we say that we have a stateful function, we really mean that every time we
``call'' the function, we get back an ``updated'' function with the new state that behaves
differently when ``called'').

To put it in terms of \texttt{settableAuto}:

\begin{itemize}
\tightlist
\item
  The ``input'' of \texttt{settableAuto} is our \texttt{Maybe\ Int} by which we specify to reset or
  allow to increment by one as normal.
\item
  The ``output'' of \texttt{settableAuto} is the ``head'' of the \texttt{ACons} that is returned ---
  the \texttt{x}, \texttt{y}, etc. It's the \texttt{Int}, the counter.
\item
  The ``state'' of \texttt{settableAuto} is, in essense, the \texttt{n} of
  \texttt{settableCounterFrom\ n}. It's the internal value by which the behavior is determined. The
  behavior of \texttt{runAuto} depends on the \texttt{n} --- it either yields \texttt{n} itself and
  increments \texttt{n}, or ignores it.
\end{itemize}

\subsubsection{The opaque state}\label{the-opaque-state}

It's a little tricky because the ``output'' and the ``state'' in our example function seem to be
exactly the same (just like for \texttt{myStream}), but let's whip up a quick example where it's a
little more obvious that the state and the output are different things, and that the state is
completely opaque and encapsulated.

\begin{Shaded}
\begin{Highlighting}[]
\OtherTok{isEvenAuto ::} \DataTypeTok{Auto} \NormalTok{(}\DataTypeTok{Maybe} \DataTypeTok{Int}\NormalTok{) }\DataTypeTok{Bool}
\NormalTok{isEvenAuto }\FunctionTok{=} \NormalTok{isEvenAuto }\DecValTok{1}
  \KeywordTok{where}
\OtherTok{    isEvenAutoFrom ::} \DataTypeTok{Int} \OtherTok{->} \DataTypeTok{Auto} \NormalTok{(}\DataTypeTok{Maybe} \DataTypeTok{Int}\NormalTok{) }\DataTypeTok{Bool}
    \NormalTok{isEvenAutoFrom n }\FunctionTok{=} \DataTypeTok{ACons} \FunctionTok{$} \NormalTok{\textbackslash{}reset }\OtherTok{->}
      \KeywordTok{let} \NormalTok{c }\FunctionTok{=} \NormalTok{fromMaybe n reset}
      \KeywordTok{in}  \NormalTok{( even c, isEvenAutoFrom (c }\FunctionTok{+} \DecValTok{1}\NormalTok{) )}
\end{Highlighting}
\end{Shaded}

So \texttt{isEvenAuto} is the same as \texttt{settableCounterFrom}, except instead of
``yielding''/``outputting'' \texttt{n}, it outputs \texttt{even\ n} --- \texttt{True} if \texttt{n}
is even and \texttt{False} if \texttt{n} is odd.

Here is a demonstration of its behavior ---

\begin{verbatim}
λ: testAuto isEvenAuto  [ Nothing, Nothing, Just 10
                        , Nothing, Nothing, Just (-1)
                        , Nothing ]
[False,True,True,False,True]
\end{verbatim}

Note that there is in general really no way to ever access the \texttt{n} internally. It is
completely sealed off from the world, except by explicit design. Here, we choose to only ``offer'' a
way to ``set'' it using our input.

Now it the three distinct concepts --- the input, output, and state --- should be clear.

\begin{itemize}
\tightlist
\item
  The ``input'' again is a \texttt{Maybe\ Int} where we can choose to reset the march of the state.
\item
  The ``output'' here is now a \texttt{Bool} that says whether or not the internal state is even.
\item
  The ``state'' here is still that \texttt{n} (an \texttt{Int}), and was the same as in the last
  example. But here it is more clear that the state is inaccessible in general. We can only modify
  it in ways that the Auto \emph{itself} allows our ``input'' (in this case, a setter) to modify it.
  And we certainly can't arbitrarily ``read'' it.
\end{itemize}

\subsection{Autos are nice!}\label{autos-are-nice}

It should be clear now that Autos are a more or less straightforward extension of Streams.

In fact, you might see that every \texttt{Stream\ b} is equivalent to \texttt{Auto\ ()\ b}, where
the input is always unit.

Autos correspond loosely to the mathematical \href{http://en.wikipedia.org/wiki/Mealy_machine}{Mealy
machine}, but again with technically potentially infinitely many possible states, input, and output
values.

Now, we have a way to model behaviors that can somehow interact with the outside world.

\subsubsection{More on state}\label{more-on-state}

Another short diversion with concerning state! As we have seen, Auto's carry all of the
hidden-internal-state features of Streams. The type of an Auto (\texttt{Auto\ a\ b}) reveals the
type of the ``input'' and the ``ouput''\ldots{}but it never reveals nor fixes the type of the
``state''. The type of the state is not only unknown, by possibly dynamically changing over the
course of the Auto's progression.

What Auto offers over Stream is then a way for the outside world to access and modify the state
\emph{if the Auto wants it to}. Now, we can design Autos that, like we have seen with
\texttt{settableAuto}, we can offer limited ways to allow the world to modify the state on our own
terms.

\subsection{The Accumulator}\label{the-accumulator}

Let's try our hand at another Auto, but instead of looking at things as an influencable and
eternally marching stream, we're going to try to look at things as a function with state that
affects its output.

How about an Auto that ``accumulates'' and ``sums up'' all of its incoming inputs, starting at 0?
More correctly, an Auto that, given any int, ``returns'' the sum of that int with all of the
previous ints it has received in its lifetime.

\begin{Shaded}
\begin{Highlighting}[]
\OtherTok{summer ::} \DataTypeTok{Auto} \DataTypeTok{Int} \DataTypeTok{Int}
\NormalTok{summer }\FunctionTok{=} \NormalTok{sumFrom }\DecValTok{0}
  \KeywordTok{where}
\OtherTok{    sumFrom ::} \DataTypeTok{Int} \OtherTok{->} \DataTypeTok{Auto} \DataTypeTok{Int} \DataTypeTok{Int}
    \NormalTok{sumFrom n }\FunctionTok{=} \DataTypeTok{ACons} \FunctionTok{$} \NormalTok{\textbackslash{}input }\OtherTok{->}
      \KeywordTok{let} \NormalTok{s }\FunctionTok{=} \NormalTok{n }\FunctionTok{+} \NormalTok{input}
      \KeywordTok{in}  \NormalTok{( s , sumFrom s )}
\end{Highlighting}
\end{Shaded}

\begin{Shaded}
\begin{Highlighting}[]
\NormalTok{λ}\FunctionTok{:} \KeywordTok{let} \NormalTok{(out1, auto1) }\FunctionTok{=} \NormalTok{runAuto summer }\DecValTok{10}
\NormalTok{λ}\FunctionTok{:} \NormalTok{out1}
\DecValTok{10}
\NormalTok{λ}\FunctionTok{:} \KeywordTok{let} \NormalTok{(out2, auto2) }\FunctionTok{=} \NormalTok{runAuto auto1 }\DecValTok{3}
\NormalTok{λ}\FunctionTok{:} \NormalTok{out1}
\DecValTok{13}
\NormalTok{λ}\FunctionTok{:} \NormalTok{testAuto_ auto2 [}\DecValTok{15}\NormalTok{,}\FunctionTok{-}\DecValTok{17}\NormalTok{,}\DecValTok{6}\NormalTok{,}\DecValTok{0}\NormalTok{,}\FunctionTok{-}\DecValTok{1}\NormalTok{]}
\NormalTok{[}\DecValTok{28}\NormalTok{,}\DecValTok{11}\NormalTok{,}\DecValTok{17}\NormalTok{,}\DecValTok{17}\NormalTok{,}\DecValTok{16}\NormalTok{]}
\end{Highlighting}
\end{Shaded}

\begin{itemize}
\tightlist
\item
  The ``input'' is our incoming \texttt{Int} --- 10, 3, 15, -17, etc.
\item
  The ``output'' is the accumulated sum/integral -- 10, 13, 28, 11, etc.
\item
  The ``state'' in this case is the accumulator, which in this case stays in sync with the output.
  But remember that this is not the case in general.
\end{itemize}

Just for kicks, let's generalize this and make an Auto version of \texttt{foldl}: give us an
operator and an initial value, and we'll ``fold up'' all of our inputs.

\begin{Shaded}
\begin{Highlighting}[]
\OtherTok{autoFold ::} \NormalTok{forall a b}\FunctionTok{.} \NormalTok{(b }\OtherTok{->} \NormalTok{a }\OtherTok{->} \NormalTok{b) }\OtherTok{->} \NormalTok{b }\OtherTok{->} \DataTypeTok{Auto} \NormalTok{a b}
\NormalTok{autoFold op init }\FunctionTok{=} \NormalTok{foldFrom init}
  \KeywordTok{where}
\OtherTok{    foldFrom ::} \NormalTok{b }\OtherTok{->} \DataTypeTok{Auto} \NormalTok{a b}
    \NormalTok{foldFrom x }\FunctionTok{=} \DataTypeTok{ACons} \FunctionTok{$} \NormalTok{\textbackslash{}input }\OtherTok{->}
      \KeywordTok{let} \NormalTok{y }\FunctionTok{=} \NormalTok{x }\OtherTok{`op`} \NormalTok{input}
      \KeywordTok{in}  \NormalTok{( y, foldFrom y )}
\end{Highlighting}
\end{Shaded}

(the \texttt{forall} is used with the
\href{http://www.haskell.org/haskellwiki/Scoped_type_variables}{Scoped Type Variables} extension to
let us say that the \texttt{b} we mention in the type of \texttt{foldFrom} is the same as the
\texttt{b} in the type of \texttt{autoFold}. If we leave off the type signature of
\texttt{foldFrom}, this is not necessary)

Note that \texttt{summer} then is just \texttt{autoFold\ (+)\ 0}.

You can probably imagine lots of different folds you can turn into \texttt{autoFold}s\ldots{}and
indeed a lot of practical Autos are just \texttt{autoFold}s. Here are some cute ones:

\begin{Shaded}
\begin{Highlighting}[]
\OtherTok{accumulateIntoList ::} \DataTypeTok{Auto} \NormalTok{a [a]}
\NormalTok{accumulateIntoList }\FunctionTok{=} \NormalTok{autoFold (flip (}\FunctionTok{:}\NormalTok{)) []}

\OtherTok{productor ::} \DataTypeTok{Num} \NormalTok{a }\OtherTok{=>} \DataTypeTok{Auto} \NormalTok{a a}
\NormalTok{productor }\FunctionTok{=} \NormalTok{autoFold (}\FunctionTok{*}\NormalTok{) }\DecValTok{1}

\OtherTok{accumulateStrings ::} \DataTypeTok{Auto} \DataTypeTok{String} \DataTypeTok{String}
\NormalTok{accumulateStrings }\FunctionTok{=} \NormalTok{autoFold (}\FunctionTok{++}\NormalTok{) }\StringTok{""}

\OtherTok{monoidAccum ::} \DataTypeTok{Monoid} \NormalTok{a }\OtherTok{=>} \DataTypeTok{Auto} \NormalTok{a a}
\NormalTok{monoidAccum }\FunctionTok{=} \NormalTok{autoFold mappend mempty}
\end{Highlighting}
\end{Shaded}

Cool, huh?

\subsubsection{Parallels with list folds}\label{parallels-with-list-folds}

Let's look very carefully at a comparison between the type signature of Prelude's \texttt{foldl} and
the type signature of \texttt{autoFold}:

\begin{Shaded}
\begin{Highlighting}[]
\NormalTok{foldl}\OtherTok{      ::} \NormalTok{(b }\OtherTok{->} \NormalTok{a }\OtherTok{->} \NormalTok{b) }\OtherTok{->} \NormalTok{b }\OtherTok{->} \NormalTok{([a] }\OtherTok{->} \NormalTok{b)}
\OtherTok{autoFold   ::} \NormalTok{(b }\OtherTok{->} \NormalTok{a }\OtherTok{->} \NormalTok{b) }\OtherTok{->} \NormalTok{b }\OtherTok{->}  \DataTypeTok{Auto} \NormalTok{a b}
\end{Highlighting}
\end{Shaded}

Hm. Let's do some rearranging. Remember that in Haskell, \texttt{(-\textgreater{})} is just an infix
type operator. So we can always rewrite \texttt{a\ -\textgreater{}\ b} as
\texttt{(-\textgreater{})\ a\ b}

\begin{Shaded}
\begin{Highlighting}[]
\NormalTok{foldl}\OtherTok{      ::} \NormalTok{(b }\OtherTok{->} \NormalTok{a }\OtherTok{->} \NormalTok{b) }\OtherTok{->} \NormalTok{b }\OtherTok{->} \NormalTok{( (}\OtherTok{->}\NormalTok{) [a] b )}
\OtherTok{autoFold   ::} \NormalTok{(b }\OtherTok{->} \NormalTok{a }\OtherTok{->} \NormalTok{b) }\OtherTok{->} \NormalTok{b }\OtherTok{->} \NormalTok{( }\DataTypeTok{Auto}  \NormalTok{a  b )}
\end{Highlighting}
\end{Shaded}

Let's get rid of some of the points, too:

\begin{Shaded}
\begin{Highlighting}[]
\NormalTok{foldl    op}\OtherTok{ init  ::} \NormalTok{(}\OtherTok{->}\NormalTok{) [a] b}
\NormalTok{autoFold op}\OtherTok{ init  ::} \DataTypeTok{Auto}  \NormalTok{a  b}
\end{Highlighting}
\end{Shaded}

So both \texttt{foldl} and \texttt{autoFold} have very similar behaviors:

Give \texttt{foldl} or \texttt{autoFold} an \emph{accumulating function} and an \emph{initial
value}, and they return \emph{a new ``function thing''}.

For \texttt{foldl\ op\ initial}, this ``function thing'' takes a \emph{list} of \texttt{a} values
and condenses them into a \texttt{b} value.

For \texttt{autoFold\ op\ initial}, the ``function thing'' takes \emph{one} \texttt{a} value and
returns a \texttt{b} value based on the previous \texttt{a}'s it has seen.

The main point here is that \texttt{autoFold} is a sort of ``function'' in a way\ldots{}just like
the others before it. It's a\ldots{}.``function-like thing''.

\subparagraph{Aside}\label{aside}

Here is a quick diversion, if you're up for it. This doesn't really have too much to do with the
rest of the post, but it'll help you test your intuition a bit with Autos.

As an exercise, compare (and contrast) these three functions of identical type signatures:

\begin{Shaded}
\begin{Highlighting}[]
\NormalTok{map}\OtherTok{       f       ::} \NormalTok{[a] }\OtherTok{->} \NormalTok{[b]}
\NormalTok{scanl     op}\OtherTok{ init ::} \NormalTok{[a] }\OtherTok{->} \NormalTok{[b]}
\NormalTok{testAuto_}\OtherTok{ auto    ::} \NormalTok{[a] }\OtherTok{->} \NormalTok{[b]}
\end{Highlighting}
\end{Shaded}

(Assume that \texttt{scanl} does not include the initial accumulator\ldots{}that is, we are really
talking about \texttt{drop\ 1\ .\ scanl\ op\ init})

Compare what they do conceptually. Then, for fun, try implementing some of them in terms of the
other. Which re-implementations are possible? Which ones arent?

\subsection{More Auto examples}\label{more-auto-examples}

Here is an Auto that outputs a rolling average of the values it accumulates, with a given window
size.

\begin{Shaded}
\begin{Highlighting}[]
\OtherTok{rollingAverage ::} \NormalTok{forall a}\FunctionTok{.} \DataTypeTok{Fractional} \NormalTok{a}
    \OtherTok{=>} \DataTypeTok{Int}          \CommentTok{-- length of the window}
    \OtherTok{->} \DataTypeTok{Auto} \NormalTok{a a     }\CommentTok{-- an Auto taking an `a` and returning an average `a`}
\NormalTok{rollingAverage window }\FunctionTok{=} \NormalTok{roll []}
  \KeywordTok{where}
\OtherTok{    roll ::} \NormalTok{[a] }\OtherTok{->} \DataTypeTok{Auto} \NormalTok{a a}
    \NormalTok{roll xs }\FunctionTok{=} \DataTypeTok{ACons} \FunctionTok{$} \NormalTok{\textbackslash{}val }\OtherTok{->}
      \KeywordTok{let} \NormalTok{xs' }\FunctionTok{=} \NormalTok{take window }\FunctionTok{$} \NormalTok{val}\FunctionTok{:}\NormalTok{xs  }\CommentTok{-- pop on the new value, drop all}
                                      \CommentTok{--   values past the window}
          \NormalTok{ave }\FunctionTok{=} \NormalTok{sum xs' }\FunctionTok{/} \NormalTok{genericLength xs'  }\CommentTok{-- the average}
      \KeywordTok{in}  \NormalTok{( ave, roll xs' )}
\end{Highlighting}
\end{Shaded}

\begin{Shaded}
\begin{Highlighting}[]
\NormalTok{λ}\FunctionTok{:} \NormalTok{testAuto_ (rollingAverage }\DecValTok{4}\NormalTok{) [}\DecValTok{2}\NormalTok{,}\DecValTok{8}\NormalTok{,}\DecValTok{4}\NormalTok{,}\DecValTok{5}\NormalTok{,}\DecValTok{1}\NormalTok{,}\DecValTok{8}\NormalTok{,}\DecValTok{3}\NormalTok{,}\DecValTok{5}\NormalTok{,}\DecValTok{1}\NormalTok{,}\DecValTok{1}\NormalTok{,}\DecValTok{8}\NormalTok{,}\DecValTok{3}\NormalTok{,}\DecValTok{5}\NormalTok{,}\DecValTok{9}\NormalTok{,}\DecValTok{2}\NormalTok{]}
\NormalTok{[}\FloatTok{2.0} \NormalTok{,}\FloatTok{5.0} \NormalTok{,}\FloatTok{4.67}\NormalTok{,}\FloatTok{4.75}\NormalTok{,}\FloatTok{4.5}
\NormalTok{,}\FloatTok{4.5} \NormalTok{,}\FloatTok{4.25}\NormalTok{,}\FloatTok{4.25}\NormalTok{,}\FloatTok{4.25}\NormalTok{,}\FloatTok{2.5}
\NormalTok{,}\FloatTok{3.75}\NormalTok{,}\FloatTok{3.25}\NormalTok{,}\FloatTok{4.25}\NormalTok{,}\FloatTok{6.25}\NormalTok{,}\FloatTok{4.75}\NormalTok{]}
\end{Highlighting}
\end{Shaded}

Here is an Auto that is always \texttt{False}\ldots{}except whenever it receives a signal matching a
given predicate (if it is ``triggered''), it remains \texttt{True} for a specified amount of time.

\begin{Shaded}
\begin{Highlighting}[]
\OtherTok{onFor ::} \NormalTok{forall a}\FunctionTok{.}
     \NormalTok{(a }\OtherTok{->} \DataTypeTok{Bool}\NormalTok{)  }\CommentTok{-- test to see if an input 'triggers'}
  \OtherTok{->} \DataTypeTok{Int}          \CommentTok{-- amount of time to stay True for}
  \OtherTok{->} \DataTypeTok{Auto} \NormalTok{a }\DataTypeTok{Bool}  \CommentTok{-- An Auto that takes an `a` and returns a `Bool`}
\NormalTok{onFor p hold }\FunctionTok{=} \NormalTok{wait}
  \KeywordTok{where}
\OtherTok{    wait ::} \DataTypeTok{Auto} \NormalTok{a }\DataTypeTok{Bool}                 \CommentTok{-- the "waiting" state}
    \NormalTok{wait }\FunctionTok{=} \DataTypeTok{ACons} \FunctionTok{$} \NormalTok{\textbackslash{}input }\OtherTok{->}
      \KeywordTok{if} \NormalTok{p input                        }\CommentTok{-- if triggered,}
        \KeywordTok{then} \NormalTok{(}\DataTypeTok{True}\NormalTok{, countdown (hold}\FunctionTok{-}\DecValTok{1}\NormalTok{)) }\CommentTok{-- jump to "countdown" state}
        \KeywordTok{else} \NormalTok{(}\DataTypeTok{False}\NormalTok{, wait)              }\CommentTok{-- otherwise, stay waiting}

\OtherTok{    countdown ::} \DataTypeTok{Int} \OtherTok{->} \DataTypeTok{Auto} \NormalTok{a }\DataTypeTok{Bool}     \CommentTok{-- the "countdown" state}
    \NormalTok{countdown n }\FunctionTok{=} \DataTypeTok{ACons} \FunctionTok{$} \NormalTok{\textbackslash{}input }\OtherTok{->}
      \KeywordTok{if} \NormalTok{p input                        }\CommentTok{-- if re-triggered}
        \KeywordTok{then} \NormalTok{(}\DataTypeTok{True}\NormalTok{, countdown (hold}\FunctionTok{-}\DecValTok{1}\NormalTok{)) }\CommentTok{-- countdown all over again}
        \KeywordTok{else}
          \KeywordTok{if} \NormalTok{n }\FunctionTok{==} \DecValTok{1}
            \KeywordTok{then} \NormalTok{(}\DataTypeTok{False}\NormalTok{, wait)          }\CommentTok{-- If counted down, go wait again}
            \KeywordTok{else} \NormalTok{(}\DataTypeTok{True}\NormalTok{, countdown (n}\FunctionTok{-}\DecValTok{1}\NormalTok{))  }\CommentTok{-- otherwise, count down.}
\end{Highlighting}
\end{Shaded}

\begin{Shaded}
\begin{Highlighting}[]
\NormalTok{λ}\FunctionTok{:} \FunctionTok{:}\NormalTok{t onFor even }\DecValTok{3}
\NormalTok{onFor even }\DecValTok{3}\OtherTok{ ::} \DataTypeTok{Auto} \DataTypeTok{Int} \DataTypeTok{Bool}
\NormalTok{λ}\FunctionTok{:} \NormalTok{testAuto_ (onFor even }\DecValTok{3}\NormalTok{) [}\DecValTok{1}\NormalTok{,}\DecValTok{1}\NormalTok{,}\DecValTok{2}\NormalTok{,}\DecValTok{1}\NormalTok{,}\DecValTok{1}\NormalTok{,}\DecValTok{1}\NormalTok{,}\DecValTok{1}\NormalTok{,}\DecValTok{4}\NormalTok{,}\DecValTok{1}\NormalTok{,}\DecValTok{6}\NormalTok{,}\DecValTok{1}\NormalTok{,}\DecValTok{1}\NormalTok{,}\DecValTok{1}\NormalTok{,}\DecValTok{1}\NormalTok{]}
\NormalTok{[ }\DataTypeTok{False}\NormalTok{, }\DataTypeTok{False}\NormalTok{, }\DataTypeTok{True} \NormalTok{, }\DataTypeTok{True}\NormalTok{,}\DataTypeTok{True}
\NormalTok{, }\DataTypeTok{False}\NormalTok{, }\DataTypeTok{True} \NormalTok{, }\DataTypeTok{True} \NormalTok{, }\DataTypeTok{True}\NormalTok{,}\DataTypeTok{True}
\NormalTok{, }\DataTypeTok{True} \NormalTok{, }\DataTypeTok{False}\NormalTok{, }\DataTypeTok{False} \NormalTok{]}
\end{Highlighting}
\end{Shaded}

Here is an auto that keeps a Map as its state, and takes as input a data type
\texttt{data\ Command\ k\ v\ =\ Insert\ k\ v\ \textbar{}\ Lookup\ k\ \textbar{}\ Delete\ k} allows
you to either insert, lookup, or delete an item. However, the map has a ``maximum size''. If you
insert past the maximum size, the insert will fail and you'll receive a \texttt{Nothing}. Otherwise,
you'll receive a \texttt{Just\ val} with the value you just inserted. Successful lookups will return
\texttt{Just\ val}, or \texttt{Nothing} on a failure. Same for deletions.

\begin{Shaded}
\begin{Highlighting}[]
\KeywordTok{import qualified} \DataTypeTok{Data.Map.Strict} \KeywordTok{as} \DataTypeTok{Map}

\KeywordTok{data} \DataTypeTok{Command} \NormalTok{k v }\FunctionTok{=} \DataTypeTok{Insert} \NormalTok{k v }\FunctionTok{|} \DataTypeTok{Lookup} \NormalTok{k }\FunctionTok{|} \DataTypeTok{Delete} \NormalTok{k}

\OtherTok{autoMap ::} \NormalTok{forall k v}\FunctionTok{.} \DataTypeTok{Ord} \NormalTok{k}
    \OtherTok{=>} \DataTypeTok{Int}              \CommentTok{-- the maximum capacity of the map}
    \OtherTok{->} \DataTypeTok{Auto} \NormalTok{(}\DataTypeTok{Command} \NormalTok{k v) (}\DataTypeTok{Maybe} \NormalTok{v)}
\NormalTok{autoMap cap }\FunctionTok{=} \NormalTok{go Map.empty}
  \KeywordTok{where}
\OtherTok{    go ::} \DataTypeTok{Map.Map} \NormalTok{k v }\OtherTok{->} \DataTypeTok{Auto} \NormalTok{(}\DataTypeTok{Command} \NormalTok{k v) (}\DataTypeTok{Maybe} \NormalTok{v)}
    \NormalTok{go m }\FunctionTok{=} \DataTypeTok{ACons} \FunctionTok{$} \NormalTok{\textbackslash{}command }\OtherTok{->}
      \KeywordTok{case} \NormalTok{command }\KeywordTok{of}
        \DataTypeTok{Insert} \NormalTok{key val }\OtherTok{->}
          \KeywordTok{if} \NormalTok{Map.size m }\FunctionTok{>=} \NormalTok{cap }\FunctionTok{&&} \NormalTok{key }\OtherTok{`Map.notMember`} \NormalTok{m}
            \KeywordTok{then}
              \NormalTok{( }\DataTypeTok{Nothing}\NormalTok{, go m )                 }\CommentTok{-- Map is full, no go!}
            \KeywordTok{else}
              \KeywordTok{let} \NormalTok{m' }\FunctionTok{=} \NormalTok{Map.insert key val m     }\CommentTok{-- go for it!}
              \KeywordTok{in}  \NormalTok{( }\DataTypeTok{Just} \NormalTok{val, go m' )}
        \DataTypeTok{Lookup} \NormalTok{key }\OtherTok{->}
          \NormalTok{( key }\OtherTok{`Map.lookup`} \NormalTok{m, go m )}
        \DataTypeTok{Delete} \NormalTok{key }\OtherTok{->}
          \KeywordTok{let} \NormalTok{result  }\FunctionTok{=} \NormalTok{key }\OtherTok{`Map.lookup`} \NormalTok{m}
              \NormalTok{m'      }\FunctionTok{=} \NormalTok{key }\OtherTok{`Map.delete`} \NormalTok{m}
          \KeywordTok{in}  \NormalTok{( result, go m' )}
\end{Highlighting}
\end{Shaded}

\begin{Shaded}
\begin{Highlighting}[]
\NormalTok{λ}\FunctionTok{:} \NormalTok{testAuto_ (autoMap }\DecValTok{3}\NormalTok{)}
  \FunctionTok{|}  \NormalTok{[ }\DataTypeTok{Insert} \StringTok{"hello"} \DecValTok{7}
  \FunctionTok{|}  \NormalTok{, }\DataTypeTok{Insert} \StringTok{"world"} \DecValTok{10}
  \FunctionTok{|}  \NormalTok{, }\DataTypeTok{Insert} \StringTok{"foo"} \DecValTok{12}
  \FunctionTok{|}  \NormalTok{, }\DataTypeTok{Insert} \StringTok{"bar"} \DecValTok{15}
  \FunctionTok{|}  \NormalTok{, }\DataTypeTok{Delete} \StringTok{"baz"}
  \FunctionTok{|}  \NormalTok{, }\DataTypeTok{Delete} \StringTok{"world"}
  \FunctionTok{|}  \NormalTok{, }\DataTypeTok{Insert} \StringTok{"haskell"} \DecValTok{19}
  \FunctionTok{|}  \NormalTok{, }\DataTypeTok{Lookup} \StringTok{"world"}
  \FunctionTok{|}  \NormalTok{, }\DataTypeTok{Lookup} \StringTok{"hello"}
  \FunctionTok{|}  \NormalTok{]}
\NormalTok{[ }\DataTypeTok{Just} \DecValTok{7} \NormalTok{, }\DataTypeTok{Just} \DecValTok{10}\NormalTok{, }\DataTypeTok{Just} \DecValTok{12}
\NormalTok{, }\DataTypeTok{Nothing}\NormalTok{, }\DataTypeTok{Nothing}\NormalTok{, }\DataTypeTok{Just} \DecValTok{10}
\NormalTok{, }\DataTypeTok{Just} \DecValTok{19}\NormalTok{, }\DataTypeTok{Nothing}\NormalTok{, }\DataTypeTok{Just} \DecValTok{7}  \NormalTok{]}
\end{Highlighting}
\end{Shaded}

The main thing to note here is that you get to completely encapsulate your ``state'', and allow it
only to be ``modified'' or ``viewed'' under your own terms. In OOP terms, it is like exposing only a
few public methods to modify your private state with discrimination. If you were passed an
\texttt{autoMap} with items already inside, you would have no way to have full ``access'' to the map
--- you would never be able to perform general operations (such as getting a list of all of the
keys).

\section{\texorpdfstring{``Function Things''}{Function Things}}\label{function-things}

Anyways, back to our main point of emphasis:

\emph{Autos are function-like things}.

They are functions\ldots{}with state.

Let's do an analysis for \texttt{isEvenAuto} like the one we did with \texttt{foldAuto}. Our
``input'' was a \texttt{Maybe\ Int} and our ``output'' was a \texttt{Bool}.

You can think of \texttt{isEvenAuto} as a ``function thing'' from \texttt{Maybe\ Int} to
\texttt{Bool}.

Here's another function from \texttt{Maybe\ Int} to \texttt{Bool}: (I'm going to be using the prefix
form of \texttt{(-\textgreater{})} a lot now for the ``final'' return functions)

\begin{Shaded}
\begin{Highlighting}[]
\OtherTok{maybeIsEven ::} \NormalTok{(}\OtherTok{->}\NormalTok{) (}\DataTypeTok{Maybe} \DataTypeTok{Int}\NormalTok{) }\DataTypeTok{Bool}
\NormalTok{maybeIsEven }\FunctionTok{=} \NormalTok{even }\FunctionTok{.} \NormalTok{fromMaybe }\DecValTok{1}
\end{Highlighting}
\end{Shaded}

\texttt{maybeIsEven} returns \texttt{True} when value inside the \texttt{Just} is even, or
\texttt{False} if the value is odd or it's a \texttt{Nothing}.

Compare that type signature to that of our \texttt{isEvenAuto}

\begin{Shaded}
\begin{Highlighting}[]
\OtherTok{maybeIsEven ::} \NormalTok{(}\OtherTok{->}\NormalTok{) (}\DataTypeTok{Maybe} \DataTypeTok{Int}\NormalTok{) }\DataTypeTok{Bool}
\OtherTok{isEvenAuto  ::} \DataTypeTok{Auto} \NormalTok{(}\DataTypeTok{Maybe} \DataTypeTok{Int}\NormalTok{) }\DataTypeTok{Bool}
\end{Highlighting}
\end{Shaded}

\texttt{maybeIsEven} and \texttt{isEvenAuto} are \emph{both} ``function-like things''. But whereas
\texttt{maybeIsEven} is ``memoryless'' (it's the same every time you call it), \texttt{isEvenAuto}
\emph{has memory} --- it returns a different Boolean based on its history.

Contrast this with a Stream, which as we have seen is just an \texttt{Auto\ ()\ b}. Streams are then
``function like things'' analgous to some \texttt{(-\textgreater{})\ ()\ b}, or
\texttt{()\ -\textgreater{}\ b}. We can call functions like \texttt{()\ -\textgreater{}\ b}
``constants'', or ``producers''. They are the same every time you call them. Streams, however,
``return'' a potentially different \texttt{b} value every time they are ``called''. So, just like an
Auto is a ``function'' that has memory, a Stream is like a \emph{constant} that has memory. A
stateful generator. A ``constant'' that returns something different every time you ask for it.

You should be able to guess that, after vaguely using the phrase ``function things'' several
times\ldots{}I'm going to surprise you all with the revelation that these ``function things'' have a
name! And maybe even\ldots{}a typeclass?

\section{Onward}\label{onward}

So far we haven't really made too convincing of an argument for the advantages of using machines
(like Auto and the related Wire). Yeah, they provide encapsulation and a changing state\ldots{}but
these things come for free in most good Object-Oriented Programming languages. So what gives?

As it turns out, as we suggested before, Autos are much more ``composable'' than the objects of OOP.
That is because, at their heart, they are just functions. And what do functions do best? They
compose! Complex object built seamlessly from simpler ones.

Now, I haven't really been able to back this up so far. But in the next post, as we explore more the
function-like nature of these things, we will be able to witness the full power of machine
composition. And we'll even be able to re-implement \emph{many} of the complex machines of this post
with compositions of smaller, simpler Autos.

\end{document}
